\documentclass[./main.tex]{subfiles}

\begin{document}

\section{Product}

\begin{definition}[Product]
  Let $\mathcal{C}$ a category and $A, B \in \mathcal{C}$,
  $(A \times B, \pi_0, \pi_1)$ forms a product of $A$ and $B$
  where $A \times B \in \mathcal{C}$, $\pi_0 : A \times B \rightarrow A$ 
  and $\pi_1 : A \times B \rightarrow B$,
  if for any $X \in \mathcal{C}$ with $\pi^\prime_0 : X \rightarrow A$ and
  $\pi^\prime_1 : X \rightarrow B$,
  there is a unique arrow $u : X \rightarrow A \times B$ such that the following
  diagram commutes:

  \begin{center}
    \begin{tikzcd}
      X 
      \arrow[rd, dashed, "u"] \arrow[rrd, bend left, "\pi^\prime_1"] 
      \arrow[ddr, bend right, "\pi^\prime_0"] \\
      & A \times B \arrow[r, "\pi_1"] \arrow[d, "\pi_0"] & B \\
      & A
    \end{tikzcd}
  \end{center}

  Furthermore, a product of $A$ and $B$ is a limit of diagram:
  \begin{center}
    \begin{tikzcd}
      (A & B)
    \end{tikzcd}
  \end{center}
\end{definition}

One may trying to show that $0 \times X \simeq 0$ by:

\begin{center}
  \begin{tikzcd}[column sep={2cm, between origins}, row sep={2cm, between origins}]
    0 \times X
    \arrow[rd, red, "\pi_0"]
    \arrow[rrrdd, bend left, red, "\pi_1"]
    \arrow[dddrr, bend right, "\pi_0"]
    \\
    & 0 
      \arrow[rd, dashed, "!"] 
      \arrow[rrd, bend left, red, "!"] 
      \arrow[ddr, bend right,"1_0"] \\
    & & 0 \times X \arrow[d, "\pi_0"] \arrow[r, "\pi_1"] & X \\
    & & 0
  \end{tikzcd}
\end{center}

However, the red triangle needs not to commutes, that is,
the arrow $\pi_0$ from $(0 \times X, \pi_0, \pi_1)$ to $(0, 1_0, !)$
may not exist.

\begin{definition}[Product of Arrow]
  Suppose $(A \times B, \pi_0, \pi_1)$ and $(C \times D, \pi_2, \pi_3)$ are two product, and
  $f : A \rightarrow C$, $g : B \rightarrow D$.
  The product of arrow $f \times g$ is a unique arrow from $A \times B$ to $C \times D$
  such that the following diagram commutes:
  % https://q.uiver.app/#q=WzAsNixbMiwwLCJBIFxcdGltZXMgQiJdLFsyLDIsIkMgXFx0aW1lcyBEIl0sWzQsMiwiRCJdLFswLDIsIkMiXSxbMCwwLCJBIl0sWzQsMCwiQiJdLFswLDUsIlxccGlfMSJdLFs1LDIsImciXSxbNCwzLCJmIiwyXSxbMSwzLCJcXHBpXzIiLDJdLFsxLDIsIlxccGlfMyJdLFswLDQsIlxccGlfMCIsMl0sWzAsMSwiZiBcXHRpbWVzIGciLDEseyJzdHlsZSI6eyJib2R5Ijp7Im5hbWUiOiJkYXNoZWQifX19XV0=
  \[\begin{tikzcd}
	A && {A \times B} && B \\
	\\
	C && {C \times D} && D
	\arrow["f"', from=1-1, to=3-1]
	\arrow["{\pi_0}"', from=1-3, to=1-1]
	\arrow["{\pi_1}", from=1-3, to=1-5]
	\arrow["{f \times g}"{description}, dashed, from=1-3, to=3-3]
	\arrow["g", from=1-5, to=3-5]
	\arrow["{\pi_2}"', from=3-3, to=3-1]
	\arrow["{\pi_3}", from=3-3, to=3-5]
\end{tikzcd}\]
\end{definition}

We may consider $\times$ a functor from $\mathcal{C} \times \mathcal{C} \rightarrow \mathcal{C}$
if $\mathcal{C}$ has any product. Then the $\times$ acts on any morphism $\langle f, g \rangle : (a, b) \rightarrow (c, d)$
is the product of arrows we showed above. The functoriality can be proved by
the uniqueness of the factor morphism.

\end{document}