\documentclass[./main.tex]{subfiles}

\begin{document}

\section{Monad}

A monad is an endofunctor $F$ equipped with:
\begin{itemize}
  \item identity $\eta : \textrm{Id} \rightarrow F$, which unsually called "pure" or "return".
  \item multiplication $\mu : F \circ F \rightarrow F$, which usually called "join".
\end{itemize}
and the identity law:
% https://q.uiver.app/#q=WzAsNSxbMCwwLCJcXG1hdGhybXtJZH0gXFxjaXJjIEYiXSxbNCwwLCJGIFxcY2lyYyBcXG1hdGhybXtJZH0iXSxbMCwyXSxbMiwwLCJGIFxcY2lyYyBGIl0sWzIsMiwiRiJdLFswLDQsIiIsMCx7InN0eWxlIjp7ImhlYWQiOnsibmFtZSI6Im5vbmUifX19XSxbMCw0LCIiLDIseyJvZmZzZXQiOi0xLCJzdHlsZSI6eyJoZWFkIjp7Im5hbWUiOiJub25lIn19fV0sWzQsMSwiIiwyLHsic3R5bGUiOnsiaGVhZCI6eyJuYW1lIjoibm9uZSJ9fX1dLFs0LDEsIiIsMix7Im9mZnNldCI6LTEsInN0eWxlIjp7ImhlYWQiOnsibmFtZSI6Im5vbmUifX19XSxbMyw0LCJcXG11IiwyXSxbMCwzLCJcXGV0YSBcXGNkb3QgRiJdLFsxLDMsIkYgXFxjZG90IFxcZXRhIiwyXV0=
\[\begin{tikzcd}
	{\mathrm{Id} \circ F} && {F \circ F} && {F \circ \mathrm{Id}} \\
	\\
	{} && F
	\arrow["{\eta \cdot F}", from=1-1, to=1-3]
	\arrow[no head, from=1-1, to=3-3]
	\arrow[shift left, no head, from=1-1, to=3-3]
	\arrow["\mu"', from=1-3, to=3-3]
	\arrow["{F \cdot \eta}"', from=1-5, to=1-3]
	\arrow[no head, from=3-3, to=1-5]
	\arrow[shift left, no head, from=3-3, to=1-5]
\end{tikzcd}\]
the associative law:
% https://q.uiver.app/#q=WzAsNixbMSwzXSxbMCwwLCIoRiBcXGNpcmMgRikgXFxjaXJjIEYiXSxbNCwwLCJGIFxcY2lyYyAoRiBcXGNpcmMgRikiXSxbMCwyLCJGIFxcY2lyYyBGIl0sWzQsMiwiRiBcXGNpcmMgRiJdLFsyLDIsIkYiXSxbMSwyLCIiLDAseyJzdHlsZSI6eyJoZWFkIjp7Im5hbWUiOiJub25lIn19fV0sWzEsMiwiIiwyLHsib2Zmc2V0IjotMSwic3R5bGUiOnsiaGVhZCI6eyJuYW1lIjoibm9uZSJ9fX1dLFsxLDMsIlxcbXUgXFxjZG90IEYiLDJdLFszLDUsIlxcbXUiLDJdLFs0LDUsIlxcbXUiXSxbMiw0LCJGIFxcY2RvdCBcXG11Il1d
\[\begin{tikzcd}
	{(F \circ F) \circ F} &&&& {F \circ (F \circ F)} \\
	\\
	{F \circ F} && F && {F \circ F} \\
	& {}
	\arrow[no head, from=1-1, to=1-5]
	\arrow[shift left, no head, from=1-1, to=1-5]
	\arrow["{\mu \cdot F}"', from=1-1, to=3-1]
	\arrow["{F \cdot \mu}", from=1-5, to=3-5]
	\arrow["\mu"', from=3-1, to=3-3]
	\arrow["\mu", from=3-5, to=3-3]
\end{tikzcd}\]

We can easily see that it is a monoid on the category of endofunctor.
If you confuse with "what the element of an endofunctor is",
try global element, just like $\eta$.

Suppose $\alpha, \beta : \textrm{Id} \rightarrow F$ are two "elements" of $F$,
then the multiplication of them will be $\mu \circ (\alpha \cdot \beta) : \textrm{Id} \rightarrow F$,
the vertical composition acts the role of function application.

\subsection{Free Monad}
Just like a list is a free monoid (on set), we also have free monad (free monoid on endofunctor):

\begin{center}
  \begin{tabular}{ |c|c| }
    \hline
    List (Free Monoid) & Free Monad \\
    \hline
    $I : \text{List} \ A$ & $\textrm{pure} : \mathrm{Id} \rightarrow \textrm{FreeMonad} \ F$ \\
    $(::) : A \otimes (\text{List} \ A) \rightarrow \text{List} \ A$ & $\textrm{free} : F \circ (\textrm{FreeMonad} \ F) \rightarrow \textrm{FreeMonad} \ F$ \\
    \hline
  \end{tabular}
\end{center}

\end{document}