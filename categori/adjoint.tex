\documentclass[./main.tex]{subfiles}

\begin{document}

\section{Adjoint}

\begin{theorem}
  Show that the hom-functor preserves limit, that is, for any $Y \in \mathcal{C}$ and
  diagram $\mathcal{D}$, we have:
  \[
    \Limit (\mathcal{C}(Y, \mathcal{D}_-)) \cong \mathcal{C}(Y, \Limit \mathcal{D})
  \]
\end{theorem}
\begin{proof}
  The idea comes from \textit{The Dao of FP} and nlab.
  
  We may consider the cone with singleton set as vertex:
% https://q.uiver.app/#q=WzAsNCxbMCw0LCJcXG1hdGhjYWx7Q30oWSwgXFxtYXRoY2Fse0R9X2kpIl0sWzIsNCwiXFxtYXRoY2Fse0N9KFksIFxcbWF0aGNhbHtEfV9qKSJdLFsxLDIsIlxcdGV4dHtMaW19IChcXG1hdGhjYWx7Q30oWSwgXFxtYXRoY2Fse0R9Xy0pKSJdLFsxLDAsIjEiXSxbMywyLCJ1Il0sWzMsMCwiXFx0ZXh0e2NvbnN0fV97cF9pfSIsMix7ImN1cnZlIjozfV0sWzMsMSwiXFx0ZXh0e2NvbnN0fV97cF9qfSIsMCx7ImN1cnZlIjotM31dLFsyLDBdLFsyLDFdLFswLDEsIlxcbWF0aGNhbHtDfShZLCBcXG1hdGhjYWx7RH1fZikiXV0=
\[\begin{tikzcd}
	& 1 \\
	\\
	& {\text{Lim} (\mathcal{C}(Y, \mathcal{D}_-))} \\
	\\
	{\mathcal{C}(Y, \mathcal{D}_i)} && {\mathcal{C}(Y, \mathcal{D}_j)}
	\arrow["u", from=1-2, to=3-2]
	\arrow["{\text{const}_{p_i}}"', curve={height=18pt}, from=1-2, to=5-1]
	\arrow["{\text{const}_{p_j}}", curve={height=-18pt}, from=1-2, to=5-3]
	\arrow[from=3-2, to=5-1]
	\arrow[from=3-2, to=5-3]
	\arrow["{\mathcal{C}(Y, \mathcal{D}_f)}", from=5-1, to=5-3]
\end{tikzcd}\]

  where $\text{const}_{p_i}$ is the function that takes a morphism $p_i \in \mathcal{C}(Y, \mathcal{D}_i)$.

  We know there is a one-to-one corresponding between $u$ and the pair
  $\langle \text{const}_{p_i} , \text{const}_{p_j} \rangle$:
  \[
    [\mathcal{J}, \CSet](\Delta_1, \mathcal{C}(Y, \mathcal{D}_-)) \cong \CSet(1, \Limit (\mathcal{C}(Y, \mathcal{D}_-)))
  \]

  or we can simplify the equation by defining $F_j = \mathcal{C}(Y, \mathcal{D}_-) : \mathcal{J} \rightarrow \CSet$.
  \[
    [\mathcal{J}, \CSet](\Delta_1, F) \cong \CSet(1, \Limit F)
  \]

  We may recall that the hom-set maps out from $1$ is isomorphic to the target,
  that is:
  \[
    \CSet(1, \Limit F) \cong \Limit F
  \]

  Similarly, the cone with $1$ as vertex is a pair of selection of $\mathcal{C}(Y, \mathcal{D}_i)$,
  it forms a cone with $Y$ as vertex:
% https://q.uiver.app/#q=WzAsMyxbMCw0LCJcXG1hdGhjYWx7RH1faSJdLFs0LDQsIlxcbWF0aGNhbHtEfV9qIl0sWzIsMCwiWSJdLFsyLDAsInBfaSIsMix7ImN1cnZlIjozfV0sWzIsMSwicF9qIiwwLHsiY3VydmUiOi0zfV0sWzAsMSwiXFxtYXRoY2Fse0R9X2YiXV0=
\[\begin{tikzcd}
	&& Y \\
	\\
	\\
	\\
	{\mathcal{D}_i} &&&& {\mathcal{D}_j}
	\arrow["{p_i}"', curve={height=18pt}, from=1-3, to=5-1]
	\arrow["{p_j}", curve={height=-18pt}, from=1-3, to=5-5]
	\arrow["{\mathcal{D}_f}", from=5-1, to=5-5]
\end{tikzcd}\]

  the diagram is indeed commute since 
% https://q.uiver.app/#q=WzAsNyxbMCwyLCJcXG1hdGhjYWx7Q30oWSwgXFxtYXRoY2Fse0R9X2kpIl0sWzIsMiwiXFxtYXRoY2Fse0N9KFksIFxcbWF0aGNhbHtEfV9qKSJdLFsxLDAsIjEiXSxbMywxLCJcXFJpZ2h0YXJyb3ciXSxbNSwwLCIxIl0sWzQsMiwicF9pIl0sWzYsMiwicF9qIl0sWzIsMCwiXFx0ZXh0e2NvbnN0fV97cF9pfSIsMix7ImN1cnZlIjoyfV0sWzAsMSwiXFxtYXRoY2Fse0N9KFksIFxcbWF0aGNhbHtEfV9mKSJdLFsyLDEsIlxcdGV4dHtjb25zdH1fe3Bfan0iLDAseyJjdXJ2ZSI6LTJ9XSxbNSw2LCJcXG1hdGhjYWx7RH1fZiBcXGNpcmMgLSIsMCx7InN0eWxlIjp7InRhaWwiOnsibmFtZSI6Im1hcHMgdG8ifX19XSxbNCw1LCJcXHRleHR7Y29uc3R9X3twX2l9IiwyLHsiY3VydmUiOjIsInN0eWxlIjp7InRhaWwiOnsibmFtZSI6Im1hcHMgdG8ifX19XSxbNCw2LCJcXHRleHR7Y29uc3R9X3twX2p9IiwwLHsiY3VydmUiOi0yLCJzdHlsZSI6eyJ0YWlsIjp7Im5hbWUiOiJtYXBzIHRvIn19fV1d
\[\begin{tikzcd}
	& 1 &&&& 1 \\
	&&& \Rightarrow \\
	{\mathcal{C}(Y, \mathcal{D}_i)} && {\mathcal{C}(Y, \mathcal{D}_j)} && {p_i} && {p_j}
	\arrow["{\text{const}_{p_i}}"', curve={height=12pt}, from=1-2, to=3-1]
	\arrow["{\text{const}_{p_j}}", curve={height=-12pt}, from=1-2, to=3-3]
	\arrow["{\text{const}_{p_i}}"', curve={height=12pt}, maps to, from=1-6, to=3-5]
	\arrow["{\text{const}_{p_j}}", curve={height=-12pt}, maps to, from=1-6, to=3-7]
	\arrow["{\mathcal{C}(Y, \mathcal{D}_f)}", from=3-1, to=3-3]
	\arrow["{\mathcal{D}_f \circ -}", maps to, from=3-5, to=3-7]
\end{tikzcd}\]

  Also we can make a pair of selection of $\mathcal{C}(Y, \mathcal{D}_-)$
  from a cone with $Y$ as vertex.

  Then the cone of $\mathcal{C}(Y, \mathcal{D}_-)$ with vertex $1$,
  is isomorphic to the cone of $\mathcal{D}_-$ with vertex $Y$:
  \[
    [\mathcal{J}, \CSet](\Delta_1, F) \cong [\mathcal{J}, \mathcal{C}](\Delta_Y, D)
  \]

  While the later one is naturally isomorphic to the limit of $\mathcal{J}_\bullet$:
  \[
    [\mathcal{J}, \mathcal{C}](\Delta_Y, D) \cong \mathcal{C}(Y, \Limit D)
  \]

  Finally, we have:
  \[
    \begin{gathered}
      \Limit F \\
      \cong \\
      \CSet(1, \Limit F) \\
      \cong \\
      \relax [\mathcal{J}, \CSet](\Delta_1, F) \\
      \cong \\ 
      \relax [\mathcal{J}, \mathcal{C}](\Delta_Y, D) \\
      \cong \\
      \mathcal{C}(Y, \Limit D)
    \end{gathered}
  \]
\end{proof}

Dually, we also have:
\[
  \Limit (\mathcal{C}(\mathcal{D}_-, Y)) \cong \mathcal{C}(\CoLimit \mathcal{D}, Y)
\]

\end{document}