\documentclass[./main.tex]{subfiles}

\begin{document}

\section{Fiber and Fibration}

I am trying to understand fiber, fibration and pullback with my stupid brain.

\subsection{Fiber}

I will use "intuitive" rather than "definition" cause I really don't
understand fiber.

\begin{intuitive}[Fiber]
  Suppose we are in a space (i.e. $\CSet$), and a mapping $f : A \rightarrow B$,
  then for some point $b \in B$, the inverse image of $b$, which is exactly $\inv{f}(b)$,
  is called a fiber of $f$ over $b$.
\end{intuitive}

We can treat a product as a pullback with apex $1$, the terminal object:

% https://q.uiver.app/#q=WzAsNSxbMiwyLCJBIFxcdGltZXMgQiJdLFs0LDIsIkIiXSxbMiw0LCJBIl0sWzQsNCwiMSJdLFswLDAsIkMiXSxbMCwyLCJcXHBpXzAiLDJdLFswLDEsIlxccGlfMSJdLFsxLDMsIiEiXSxbMiwzLCIhIiwyXSxbNCwwLCIhdSIsMSx7InN0eWxlIjp7ImJvZHkiOnsibmFtZSI6ImRhc2hlZCJ9fX1dLFs0LDEsImciLDAseyJjdXJ2ZSI6LTN9XSxbNCwyLCJmIiwyLHsiY3VydmUiOjN9XV0=
\[\begin{tikzcd}
	C \\
	\\
	&& {A \times B} && B \\
	\\
	&& A && 1
	\arrow["{!u}"{description}, dashed, from=1-1, to=3-3]
	\arrow["g", curve={height=-18pt}, from=1-1, to=3-5]
	\arrow["f"', curve={height=18pt}, from=1-1, to=5-3]
	\arrow["{\pi_1}", from=3-3, to=3-5]
	\arrow["{\pi_0}"', from=3-3, to=5-3]
	\arrow["{!}", from=3-5, to=5-5]
	\arrow["{!}"', from=5-3, to=5-5]
\end{tikzcd}\]

We can treat $A$ as the fiber against to the only point in $1$, same for $B$.
Now, what if we replace $1$ with something else?

% https://q.uiver.app/#q=WzAsNSxbMiwyLCJBIFxcdGltZXNfSSBCIl0sWzQsMiwiQiJdLFsyLDQsIkEiXSxbNCw0LCJJIl0sWzAsMCwiQyJdLFswLDIsIlxccGlfMCIsMl0sWzAsMSwiXFxwaV8xIl0sWzEsMywidCJdLFsyLDMsInMiLDJdLFs0LDAsIiF1IiwxLHsic3R5bGUiOnsiYm9keSI6eyJuYW1lIjoiZGFzaGVkIn19fV0sWzQsMSwiZyIsMCx7ImN1cnZlIjotM31dLFs0LDIsImYiLDIseyJjdXJ2ZSI6M31dXQ==
\[\begin{tikzcd}
	C \\
	\\
	&& {A \times_I B} && B \\
	\\
	&& A && I
	\arrow["{!u}"{description}, dashed, from=1-1, to=3-3]
	\arrow["g", curve={height=-18pt}, from=1-1, to=3-5]
	\arrow["f"', curve={height=18pt}, from=1-1, to=5-3]
	\arrow["{\pi_1}", from=3-3, to=3-5]
	\arrow["{\pi_0}"', from=3-3, to=5-3]
	\arrow["t", from=3-5, to=5-5]
	\arrow["s"', from=5-3, to=5-5]
\end{tikzcd}\]

For every point $i \in I$, we have fiber $A_i \subseteq A$ and $B_i \subseteq B$,
which can form a product $A_i \times B_i$. We may sum all these products, and finally
get $A \times_I B$, this is why the pullback is sometimes called \textit{fiber product}.

We can also pick certain fiber from this pullback:

% https://q.uiver.app/#q=WzAsNCxbMCwwLCJcXHZhcnBoaSJdLFsyLDAsIkEiXSxbMCwyLCIxIl0sWzIsMiwiQiJdLFswLDIsIiEiLDJdLFswLDEsImkiXSxbMSwzLCJmIl0sWzIsMywieCIsMl1d
\[\begin{tikzcd}
	\varphi && A \\
	\\
	1 && B
	\arrow["i", from=1-1, to=1-3]
	\arrow["{!}"', from=1-1, to=3-1]
	\arrow["f", from=1-3, to=3-3]
	\arrow["x"', from=3-1, to=3-3]
\end{tikzcd}\]

The morphism $x : 1 \rightarrow B$ is a global element, which "pick" an element
of $B$, then $i$ must maps $\varphi$ to the fiber of $f$ over point $x$,
which should be a injection.

The collection of fiber (the source of the morphism/the domain of the function) is called \textit{fiber bundle}.

\subsection{Fibration}

Some intuitive comes from
\href{https://guest0x0.xyz/family-and-fibration/family-and-fibration.pdf}{this article}.

\begin{intuitive}
  A fibration works like an indexed family (i.e. a function $I \rightarrow A$),
  but do it in fiber way (i.e. a function $A \rightarrow I$).
\end{intuitive}

\subsection{Base-change Functor}

~

These section is related to \daofp

We can also treat the morphism on right-hand side as a fibration, and the
bottom-left corner a base (the target of a fibration):

% https://q.uiver.app/#q=WzAsNCxbMCwwLCI/Il0sWzIsMCwiRSJdLFswLDIsIkEiXSxbMiwyLCJCIl0sWzAsMiwiIiwyLHsic3R5bGUiOnsiYm9keSI6eyJuYW1lIjoiZGFzaGVkIn19fV0sWzAsMSwiIiwwLHsic3R5bGUiOnsiYm9keSI6eyJuYW1lIjoiZGFzaGVkIn19fV0sWzEsMywicCJdLFsyLDMsImYiLDJdXQ==
\[\begin{tikzcd}
	{?} && E \\
	\\
	A && B
	\arrow[dashed, from=1-1, to=1-3]
	\arrow[dashed, from=1-1, to=3-1]
	\arrow["p", from=1-3, to=3-3]
	\arrow["f"', from=3-1, to=3-3]
\end{tikzcd}\]

Then we can treat $E \xrightarrow{p} B$ as an object in the slice category $\mathcal{C}/B$,
similarly, the left-hand side morphism an object in $\mathcal{C}/A$.
Then we can define a base-change functor $f^* : \mathcal{C}/B \rightarrow \mathcal{C}/A$
such that:

% https://q.uiver.app/#q=WzAsNCxbMCwwLCJmXiogRSJdLFsyLDAsIkUiXSxbMCwyLCJBIl0sWzIsMiwiQiJdLFswLDIsImZeKiBwIiwyLHsic3R5bGUiOnsiYm9keSI6eyJuYW1lIjoiZGFzaGVkIn19fV0sWzAsMSwiZyIsMCx7InN0eWxlIjp7ImJvZHkiOnsibmFtZSI6ImRhc2hlZCJ9fX1dLFsxLDMsInAiXSxbMiwzLCJmIiwyXV0=
\[\begin{tikzcd}
	{f^* E} && E \\
	\\
	A && B
	\arrow["g", dashed, from=1-1, to=1-3]
	\arrow["{f^* p}"', dashed, from=1-1, to=3-1]
	\arrow["p", from=1-3, to=3-3]
	\arrow["f"', from=3-1, to=3-3]
\end{tikzcd}\]
a pullback.

We denote $f^* E$ as the source of $f^* p$, it doesn't mean that $f^*$ accept a object in $\mathcal{C}$.

We need to define the action of base-change functor on the morphism of $\mathcal{C}/B$:

% https://q.uiver.app/#q=WzAsNixbMiwwLCJmXiogRSJdLFs0LDAsIkUiXSxbMiwyLCJBIl0sWzQsMiwiQiJdLFs2LDAsIkVeXFxwcmltZSJdLFswLDAsImZeKkVeXFxwcmltZSJdLFswLDIsImZeKiBwIiwyXSxbMCwxLCJnIiwxLHsic3R5bGUiOnsiYm9keSI6eyJuYW1lIjoiZGFzaGVkIn19fV0sWzEsMywicCJdLFsyLDMsImYiLDJdLFs0LDMsInBeXFxwcmltZSJdLFsxLDQsImUiLDFdLFs1LDIsImYqIHBeXFxwcmltZSIsMl0sWzAsNSwiPyIsMV0sWzUsNCwiZ15cXHByaW1lIiwwLHsiY3VydmUiOi0zLCJzdHlsZSI6eyJib2R5Ijp7Im5hbWUiOiJkYXNoZWQifX19XV0=
\[\begin{tikzcd}
	{f^*E^\prime} && {f^* E} && E && {E^\prime} \\
	\\
	&& A && B
	\arrow["{g^\prime}", curve={height=-18pt}, dashed, from=1-1, to=1-7]
	\arrow["{f* p^\prime}"', from=1-1, to=3-3]
	\arrow["{?}"{description}, from=1-3, to=1-1]
	\arrow["g"{description}, dashed, from=1-3, to=1-5]
	\arrow["{f^* p}"', from=1-3, to=3-3]
	\arrow["e"{description}, from=1-5, to=1-7]
	\arrow["p", from=1-5, to=3-5]
	\arrow["{p^\prime}", from=1-7, to=3-5]
	\arrow["f"', from=3-3, to=3-5]
\end{tikzcd}\]
Two commute triangles are the morphisms in $\mathcal{C}/A$ and $\mathcal{C}/B$.

Since $f^* E^\prime$ is a pullback of $A \rightarrow B \leftarrow E^\prime$,
it tips us that we can find the commute square below to get the morphism we want:

% https://q.uiver.app/#q=WzAsNCxbMCwwLCJmXiogRSJdLFswLDIsIkEiXSxbMiwyLCJCIl0sWzIsMCwiRV5cXHByaW1lIl0sWzAsMV0sWzEsMiwiZiIsMl0sWzMsMiwicF5cXHByaW1lIl0sWzAsM11d
\[\begin{tikzcd}
	{f^* E} && {E^\prime} \\
	\\
	A && B
	\arrow[from=1-1, to=1-3]
	\arrow[from=1-1, to=3-1]
	\arrow["{p^\prime}", from=1-3, to=3-3]
	\arrow["f"', from=3-1, to=3-3]
\end{tikzcd}\]

If we look the last diagram carefully, we can find this square commutes:

% https://q.uiver.app/#q=WzAsNSxbMCwwLCJmXiogRSJdLFsyLDAsIkUiXSxbMCwyLCJBIl0sWzIsMiwiQiJdLFs0LDAsIkVeXFxwcmltZSJdLFswLDIsImZeKiBwIiwyXSxbMCwxLCJnIiwxLHsic3R5bGUiOnsiYm9keSI6eyJuYW1lIjoiZGFzaGVkIn19fV0sWzEsMywicCIsMCx7ImNvbG91ciI6WzAsMCw2NF19LFswLDAsNjQsMV1dLFsyLDMsImYiLDJdLFs0LDMsInBeXFxwcmltZSJdLFsxLDQsImUiLDFdXQ==
\[\begin{tikzcd}
	{f^* E} && E && {E^\prime} \\
	\\
	A && B
	\arrow["g"{description}, dashed, from=1-1, to=1-3]
	\arrow["{f^* p}"', from=1-1, to=3-1]
	\arrow["e"{description}, from=1-3, to=1-5]
	\arrow["p", color={rgb,255:red,163;green,163;blue,163}, from=1-3, to=3-3]
	\arrow["{p^\prime}", from=1-5, to=3-3]
	\arrow["f"', from=3-1, to=3-3]
\end{tikzcd}\]

therefore

% https://q.uiver.app/#q=WzAsNSxbMCwwLCJmXiogRSJdLFsyLDQsIkEiXSxbNCw0LCJCIl0sWzQsMiwiRV5cXHByaW1lIl0sWzIsMiwiZipFXlxccHJpbWUiXSxbMCwxLCJmXiogcCIsMCx7ImN1cnZlIjozfV0sWzEsMiwiZiIsMl0sWzMsMiwicF5cXHByaW1lIl0sWzAsMywiZSBcXGNpcmMgZyIsMCx7ImN1cnZlIjotM31dLFs0LDMsImdeXFxwcmltZSJdLFs0LDEsImZeKiBwXlxccHJpbWUiLDJdLFswLDQsImZeKiBlIiwxLHsic3R5bGUiOnsiYm9keSI6eyJuYW1lIjoiZGFzaGVkIn19fV1d
\[\begin{tikzcd}
	{f^* E} \\
	\\
	&& {f*E^\prime} && {E^\prime} \\
	\\
	&& A && B
	\arrow["{f^* e}"{description}, dashed, from=1-1, to=3-3]
	\arrow["{e \circ g}", curve={height=-18pt}, from=1-1, to=3-5]
	\arrow["{f^* p}", curve={height=18pt}, from=1-1, to=5-3]
	\arrow["{g^\prime}", from=3-3, to=3-5]
	\arrow["{f^* p^\prime}"', from=3-3, to=5-3]
	\arrow["{p^\prime}", from=3-5, to=5-5]
	\arrow["f"', from=5-3, to=5-5]
\end{tikzcd}\]

The functoriality follows the fact that $f^* e$ is unique that makes the diagram commutes.

\end{document}