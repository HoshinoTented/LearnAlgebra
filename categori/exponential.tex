\documentclass[./main.tex]{subfiles}

\begin{document}

\section{Exponential}

\begin{definition}
  Let $\mathcal{C}$ a category.
  For any $B, C \in \mathcal{C}$, $(C^B, ev)$ forms an exponential
  where $C^B \in \mathcal{C}$ and $ev : C^B \times B \rightarrow C$,
  if for any object $A \in \mathcal{C}$ and $f : A \times B \rightarrow C$,
  there is a unique $u : A \rightarrow C^B$ such that $f = ev \circ (u \times 1_B)$.
  In other words, the follow diagram commutes.
  \begin{center}
    \begin{tikzcd}
      A \times B \arrow[dd, dashed, "u \times 1_B"] \arrow[rd, "f"] & \\
      & C \\
      C^B \times B \arrow[ru, "ev"] &
    \end{tikzcd}
  \end{center}
\end{definition}

Suppose we are in $\CSet$, we know for any function $f : A \rightarrow B$,
there is an element in some set that represents this function.
In programming perspect, we know it is an element of type $A \rightarrow B$.
What if we generialize it to other category?
In order to represent an element of some set/type/object, we can use global element,
that is, a morphism from terminal object. We denote the object that represents
the morphisms from $a$ to $b$ by $b^a$, then a morphism $f : a \rightarrow b$
should be represented by $g : 1 \rightarrow b^a$. Furthermore,
we can apply the function to some argument, that is, $\epsilon : b^a \times a \rightarrow b$,
we should require such $g$ respects this behaviour, and we get this diagram commutes:

% https://q.uiver.app/#q=WzAsMyxbMCwwLCIxIFxcdGltZXMgYSJdLFsyLDIsImIiXSxbMCwyLCJiXmEgXFx0aW1lcyBhIl0sWzAsMSwiZiJdLFsyLDEsIlxcZXBzaWxvbiIsMl0sWzAsMiwiXFxsYW5nbGUgZywgXFxtYXRocm17aWR9X2FcXHJhbmdsZSIsMl1d
\[\begin{tikzcd}
	{1 \times a} \\
	\\
	{b^a \times a} && b
	\arrow["{\langle g, \mathrm{id}_a\rangle}"', from=1-1, to=3-1]
	\arrow["f", from=1-1, to=3-3]
	\arrow["\epsilon"', from=3-1, to=3-3]
\end{tikzcd}\]

It says, if we take an element $g$ that represents $f$,
then we can apply $g$ to for any "element" of $a$, it should behave like
we apply $f$ to that "element" of $a$.

We may loose the requirement that $1$ exists, we can use any object $c$:
% https://q.uiver.app/#q=WzAsMyxbMCwwLCJjIFxcdGltZXMgYSJdLFsyLDIsImIiXSxbMCwyLCJiXmEgXFx0aW1lcyBhIl0sWzAsMSwiZiJdLFsyLDEsIlxcZXBzaWxvbiIsMl0sWzAsMiwiXFxsYW5nbGUgZywgXFxtYXRocm17aWR9X2FcXHJhbmdsZSIsMl1d
\[\begin{tikzcd}
	{c \times a} \\
	\\
	{b^a \times a} && b
	\arrow["{\langle g, \mathrm{id}_a\rangle}"', from=1-1, to=3-1]
	\arrow["f", from=1-1, to=3-3]
	\arrow["\epsilon"', from=3-1, to=3-3]
\end{tikzcd}\]
The morphism $g$ corresponds to the \textit{currying} in programming.

Then we can choose some elements of $a$ and $c$:
% https://q.uiver.app/#q=WzAsNCxbMCwyLCJjIFxcdGltZXMgYSJdLFsyLDQsImIiXSxbMCw0LCJiXmEgXFx0aW1lcyBhIl0sWzAsMCwiMSBcXHRpbWVzIDEiXSxbMCwxLCJmIl0sWzIsMSwiXFxlcHNpbG9uIiwyXSxbMCwyLCJcXGxhbmdsZSBnLCBcXG1hdGhybXtpZH1fYVxccmFuZ2xlIiwyXSxbMywwLCJcXGxhbmdsZSBpLCBqIFxccmFuZ2xlIiwyXSxbMywxLCJrIiwwLHsiY3VydmUiOi0zfV1d
\[\begin{tikzcd}
	{1 \times 1} \\
	\\
	{c \times a} \\
	\\
	{b^a \times a} && b
	\arrow["{\langle i, j \rangle}"', from=1-1, to=3-1]
	\arrow["k", curve={height=-18pt}, from=1-1, to=5-3]
	\arrow["{\langle g, \mathrm{id}_a\rangle}"', from=3-1, to=5-1]
	\arrow["f", from=3-1, to=5-3]
	\arrow["\epsilon"', from=5-1, to=5-3]
\end{tikzcd}\]
The diagram says: For a function object $g \circ i$, apply it to element $j$ by $\epsilon$,
should equivalent to we apply $f$ to them, and the result is the element $k$:
\begin{align*}
   & \epsilon \circ \langle g \circ i, j \rangle && \text{(applying through function object)}\\
  =& f \circ \langle i , j \rangle               && \text{(applying directly)} \\
  =& k                                           && \text{(the result)}
\end{align*}

Furthermore, we can observe that the morphism $f$ and the "curring" morphism $g$
is one-one corresponding. It notices us that there is a underlying adjunction:
\[
  \mathcal{C}(c \times a, b) \cong \mathcal{C}(c, b^a)
\]
It would be more clear if we rewrite them:
\[
  \begin{gathered}
    L_a c = c \times a \\
    R_a b = b^a \\
    \mathcal{C}(L_a c, b) \cong \mathcal{C}(c, R_a b)
  \end{gathered}
\]

But we need to show that they are functors. $L_a$ is a functor cause $- \times -$ is a functor
(see chapter product). For any morphism $h : s \rightarrow t$, we need to provide
a morphism $h^a : s^a \rightarrow t^a$, one reasonable choice is:
% https://q.uiver.app/#q=WzAsNCxbMCwwLCJzXmEgXFx0aW1lcyBhIl0sWzIsMiwidCJdLFswLDIsInReYSBcXHRpbWVzIGEiXSxbMiwwLCJzIl0sWzIsMSwiXFxlcHNpbG9uX3QiLDJdLFswLDIsIlxcbGFuZ2xlIGheYSwgXFxtYXRocm17aWR9X2FcXHJhbmdsZSIsMl0sWzMsMSwiZiJdLFswLDMsIlxcZXBzaWxvbl9zIl1d
\[\begin{tikzcd}
	{s^a \times a} && s \\
	\\
	{t^a \times a} && t
	\arrow["{\epsilon_s}", from=1-1, to=1-3]
	\arrow["{\langle h^a, \mathrm{id}_a\rangle}"', from=1-1, to=3-1]
	\arrow["h", from=1-3, to=3-3]
	\arrow["{\epsilon_t}"', from=3-1, to=3-3]
\end{tikzcd}\]
It is easy to check the functoriality by the uniqueness of $h^a$.
  
\end{document}