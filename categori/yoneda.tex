\documentclass[./main.tex]{subfiles}

\begin{document}

\section{Yoneda}

This chapter combines arguments from some books:
\begin{itemize}
  \item The Dao of FP
  \item The Joy of Abstraction
\end{itemize}

\begin{definition}
  $H_x = \mathcal{C}(-, x)$ and $H^x = \mathcal{C}(x, -)$.
\end{definition}

Take $H^x$ as an example, it sends $\mathcal{C}$ to $\mathbf{Set}$, the interesting part
is the mapping on morphism. For any morphism $f : a \rightarrow b$ of $\mathcal{C}$,
$H^f$ must be a mapping $\mathcal{C}(x, a) \rightarrow \mathcal{C}(x, b)$,
we can see that $g \mapsto f \circ g$ would be a choice.

We have to show that it satifies the functoriality:
\begin{itemize}
  \item $H^{id_a}(g) = id_a \circ g = g$
  \item $H^{f \circ g}(h) = (f \circ g) \circ h = f \circ (g \circ h) = H^f (g \circ h) = H^f (H^g (h)) = (H^f \circ H^g) (h)$.
\end{itemize}

Similar to $H_x$, the only difference is that $H_x$ is a contrafunctor.

Suppose $f : a \rightarrow b$ an isomorphism, we can see that $H^x$ gives
an isomorphism between two hom-sets: $\mathcal{C}(x, a)$ and $\mathcal{C}(x, b)$.

Furthermore, we can no more fix $x$, that is, make $H_{\bullet}$ (or $H^\bullet$) a functor
from $\mathcal{C}$ to $[\mathcal{C}^{op}, \textbf{Set}]$, a functor to a functor!

The problem we need to solve is that what should $H_\bullet$ do on a morphism $f : a \rightarrow b$.
Since $H_\bullet$ produce a functor, $H_f$ must produce a natural transformation
between $H_a$ and $H_b$. Suppose $x, y \in \mathcal{C}$ and $g : x \rightarrow y$,
note that $H_a$ and $H_b$ are contrafunctor, so we need to reverse the arrows!

% https://q.uiver.app/#q=WzAsNCxbMCwwLCJIX2EoeSkiXSxbMywwLCJIX2EoeCkiXSxbMCwzLCJIX2IoeSkiXSxbMywzLCJIX2IoeCkiXSxbMCwxLCJIX2EoZykiXSxbMCwyLCIoSF9mKV95IiwyXSxbMSwzLCIoSF9mKV94Il0sWzIsMywiKEhfYSkoZykiLDJdXQ==
\[\begin{tikzcd}
	{H_a(y)} &&& {H_a(x)} \\
	\\
	\\
	{H_b(y)} &&& {H_b(x)}
	\arrow["{H_a(g)}", from=1-1, to=1-4]
	\arrow["{(H_f)_y}"', from=1-1, to=4-1]
	\arrow["{(H_f)_x}", from=1-4, to=4-4]
	\arrow["{(H_b)(g)}"', from=4-1, to=4-4]
\end{tikzcd}\]

and we can unfold the definitions

% https://q.uiver.app/#q=WzAsNCxbMCwwLCJcXG1hdGhjYWx7Q30oeSwgYSkiXSxbMywwLCJcXG1hdGhjYWx7Q30oeCwgYSkiXSxbMCwzLCJcXG1hdGhjYWx7Q30oeSwgYikiXSxbMywzLCJcXG1hdGhjYWx7Q30oeCwgYikiXSxbMCwxLCJcXG1hdGhjYWx7Q30oZywgYSkiXSxbMCwyLCIoSF9mKV95IiwyXSxbMSwzLCIoSF9mKV94Il0sWzIsMywiXFxtYXRoY2Fse0N9KGcsIGIpIiwyXV0=
\[\begin{tikzcd}
	{\mathcal{C}(y, a)} &&& {\mathcal{C}(x, a)} \\
	\\
	\\
	{\mathcal{C}(y, b)} &&& {\mathcal{C}(x, b)}
	\arrow["{\mathcal{C}(g, a)}", from=1-1, to=1-4]
	\arrow["{(H_f)_y}"', from=1-1, to=4-1]
	\arrow["{(H_f)_x}", from=1-4, to=4-4]
	\arrow["{\mathcal{C}(g, b)}"', from=4-1, to=4-4]
\end{tikzcd}\]

and suppose $s \in \mathcal{C}(y, a)$, we know the top-right corner
would be $s \circ g$ since $\mathcal{C}(g, a) = - \circ g$ (same to $\mathcal{C}(g, b)$).
In order to construct an
arrow in $\mathcal{C}(y, b)$, we can pre-compose the arrow $f : a \rightarrow b$.
Then the bottom-left corner would be $f \circ s$, and the bottom-right corner
would be $(f \circ s) \circ g$ (by left-bottom path) and $f \circ (s \circ g)$ (by top-right path),
which is exactly the same!

% https://q.uiver.app/#q=WzAsNCxbMCwwLCJzIl0sWzMsMCwicyBcXGNpcmMgZyJdLFswLDMsImYgXFxjaXJjIHMiXSxbMywzLCIoZiBcXGNpcmMgcykgXFxjaXJjIGcgPSBmIFxcY2lyYyAocyBcXGNpcmMgZykiXSxbMCwxLCItIFxcY2lyYyBnIiwwLHsic3R5bGUiOnsidGFpbCI6eyJuYW1lIjoibWFwcyB0byJ9fX1dLFswLDIsImYgXFxjaXJjIC0iLDIseyJzdHlsZSI6eyJ0YWlsIjp7Im5hbWUiOiJtYXBzIHRvIn19fV0sWzEsMywiZiBcXGNpcmMgLSIsMCx7InN0eWxlIjp7InRhaWwiOnsibmFtZSI6Im1hcHMgdG8ifX19XSxbMiwzLCItIFxcY2lyYyBnIiwyLHsic3R5bGUiOnsidGFpbCI6eyJuYW1lIjoibWFwcyB0byJ9fX1dXQ==
\[\begin{tikzcd}
	s &&& {s \circ g} \\
	\\
	\\
	{f \circ s} &&& {(f \circ s) \circ g = f \circ (s \circ g)}
	\arrow["{- \circ g}", maps to, from=1-1, to=1-4]
	\arrow["{f \circ -}"', maps to, from=1-1, to=4-1]
	\arrow["{f \circ -}", maps to, from=1-4, to=4-4]
	\arrow["{- \circ g}"', maps to, from=4-1, to=4-4]
\end{tikzcd}\]

Note that the condition here $(f \circ -) \circ (- \circ g) = (- \circ g) \circ (f \circ -)$
is the naturality condition which is mentioned in $\textit{The Dao of FP}$.
A bijection between two hom-sets $\alpha_x = \mathcal{C}(x, a) \rightarrow \mathcal{C}(x, b)$
that satifies the naturality condition $\alpha_y \circ (- \circ g) = (- \circ g) \circ \alpha_x$
can retrieve the isomorphism between $a$ and $b$. This will be unsurprised
if we notice that such bijection with naturality condition
forms a natural transformation, then we can retrieve
the morphism (not isomorphism yet) from it.
The morphism becomes iso- when we know $H_\bullet$ is full and faithful
(see below and chapter \textit{functor}),


\begin{definition}
  $H_\bullet$ is called Yoneda embedding.
\end{definition}

\begin{theorem}
  Shows $H_\bullet$ is an embedding by showing it is full and faithful.
\end{theorem}
\begin{proof}
  (Full) For any $a, b \in \mathcal{C}$, suppose $\alpha : [\mathcal{C}^{op}, \textbf{Set}](H_a, H_b)$
  a morphism (natural transformation).
  Use the Yoneda trick, we have $\alpha_a : \mathcal{C}(a, a) \rightarrow \mathcal{C}(a, b)$
  and then $\alpha_a(id_a) : \mathcal{C}(a, b)$.
  As we see the definition of $H_\bullet$ on morphism, we should expect that
  $\alpha$ has form $f \circ -$ for some $f : a \rightarrow b$. But how coincident,
  we have a morphism $\alpha_a(id_a) : \mathcal{C}(a, b)$. So we claim
  $H_{\alpha_a(id_a)} = \alpha$.
  (In the other hand, if $\alpha$ has form $f \circ -$, then $\alpha_a(id_a) = f \circ id_a = f$).
  For any $x \in \mathcal{C}$, we need to show $(H_{\alpha_a(id_a)})_x = \alpha_x : \mathcal{C}(x, a) \rightarrow \mathcal{C}(x, b)$.
  So we suppose $g \in \mathcal{C}(x, a)$, then the following diagram commutes
  since $\alpha$ is natural:

  % https://q.uiver.app/#q=WzAsNCxbMCwwLCJcXG1hdGhjYWx7Q30oYSwgYSkiXSxbMywwLCJcXG1hdGhjYWx7Q30oeCwgYSkiXSxbMCwzLCJcXG1hdGhjYWx7Q30oYSwgYikiXSxbMywzLCJcXG1hdGhjYWx7Q30oeCwgYikiXSxbMCwxLCItIFxcY2lyYyBnIl0sWzAsMiwiXFxhbHBoYV9hIiwyXSxbMSwzLCJcXGFscGhhX3giXSxbMiwzLCItIFxcY2lyYyBnIiwyXV0=
  \[\begin{tikzcd}
	{\mathcal{C}(a, a)} &&& {\mathcal{C}(x, a)} \\
	\\
	\\
	{\mathcal{C}(a, b)} &&& {\mathcal{C}(x, b)}
	\arrow["{- \circ g}", from=1-1, to=1-4]
	\arrow["{\alpha_a}"', from=1-1, to=4-1]
	\arrow["{\alpha_x}", from=1-4, to=4-4]
	\arrow["{- \circ g}"', from=4-1, to=4-4]
  \end{tikzcd}\]
  Then
  % https://q.uiver.app/#q=WzAsNCxbMCwwLCJpZF9hIl0sWzMsMCwiaWRfYSBcXGNpcmMgZyA9IGciXSxbMCwzLCJcXGFscGhhX2EoaWRfYSkiXSxbMywzLCJcXGFscGhhX2EoaWRfYSkgXFxjaXJjIGcgPSBcXGFscGhhX3goZykiXSxbMCwxLCItIFxcY2lyYyBnIiwwLHsic3R5bGUiOnsidGFpbCI6eyJuYW1lIjoibWFwcyB0byJ9fX1dLFswLDIsIlxcYWxwaGFfYSIsMix7InN0eWxlIjp7InRhaWwiOnsibmFtZSI6Im1hcHMgdG8ifX19XSxbMSwzLCJcXGFscGhhX3giLDAseyJzdHlsZSI6eyJ0YWlsIjp7Im5hbWUiOiJtYXBzIHRvIn19fV0sWzIsMywiLSBcXGNpcmMgZyIsMix7InN0eWxlIjp7InRhaWwiOnsibmFtZSI6Im1hcHMgdG8ifX19XV0=
  \[\begin{tikzcd}
	{id_a} &&& {id_a \circ g = g} \\
	\\
	\\
	{\alpha_a(id_a)} &&& {\alpha_a(id_a) \circ g = \alpha_x(g)}
	\arrow["{- \circ g}", maps to, from=1-1, to=1-4]
	\arrow["{\alpha_a}"', maps to, from=1-1, to=4-1]
	\arrow["{\alpha_x}", maps to, from=1-4, to=4-4]
	\arrow["{- \circ g}"', maps to, from=4-1, to=4-4]
  \end{tikzcd}\]
  is a proof of $H_{\alpha_a(id_a)}(g) = \alpha_a(id_a) \circ g = \alpha_x(g)$.

  (Faithful) Suppose $f \circ - H_f = H_g = g \circ -$, then $f = f \circ id_a = H_f(id_a) = H_g(od_a) = g \circ id_a = g$.
\end{proof}

As we see in the proof of $H_\bullet$ is a full functor, the natural transformation
$\alpha$ at some $x$ (therefore any $x \in \mathcal{C}$) is completely determined by the value $\alpha_a(id_a)$,
cause for any $g$, we have $\alpha_a(id_a) \circ g = \alpha_x(g)$.

We may rename $H_b$ with $F$, then

% https://q.uiver.app/#q=WzAsOCxbMSwxLCJpZF9hIl0sWzQsMSwiaWRfYSBcXGNpcmMgZyA9IGciXSxbMSw0LCJcXGFscGhhX2EoaWRfYSkiXSxbNCw0LCIoRmcpKFxcYWxwaGFfYShpZF9hKSkgPSBcXGFscGhhX3goZykiXSxbMCwwLCJcXG1hdGhjYWx7Q30oYSwgYSkiXSxbNSwwLCJcXG1hdGhjYWx7Q30oeCwgYSkiXSxbMCw1LCJGIGEiXSxbNSw1LCJGeCJdLFswLDEsIi0gXFxjaXJjIGciLDAseyJzdHlsZSI6eyJ0YWlsIjp7Im5hbWUiOiJtYXBzIHRvIn19fV0sWzAsMiwiXFxhbHBoYV9hIiwyLHsic3R5bGUiOnsidGFpbCI6eyJuYW1lIjoibWFwcyB0byJ9fX1dLFsxLDMsIlxcYWxwaGFfeCIsMCx7InN0eWxlIjp7InRhaWwiOnsibmFtZSI6Im1hcHMgdG8ifX19XSxbMiwzLCJGZyIsMix7InN0eWxlIjp7InRhaWwiOnsibmFtZSI6Im1hcHMgdG8ifX19XSxbNCw1LCJcXG1hdGhjYWx7Q30oZywgYSkiXSxbNSw3LCJcXGFscGhhX3giXSxbNiw3LCJGZyIsMl0sWzQsNiwiXFxhbHBoYV9hIiwyXV0=
\[\begin{tikzcd}
	{\mathcal{C}(a, a)} &&&&& {\mathcal{C}(x, a)} \\
	& {id_a} &&& {id_a \circ g = g} \\
	\\
	\\
	& {\alpha_a(id_a)} &&& {(Fg)(\alpha_a(id_a)) = \alpha_x(g)} \\
	{F a} &&&&& Fx
	\arrow["{\mathcal{C}(g, a)}", from=1-1, to=1-6]
	\arrow["{\alpha_a}"', from=1-1, to=6-1]
	\arrow["{\alpha_x}", from=1-6, to=6-6]
	\arrow["{- \circ g}", maps to, from=2-2, to=2-5]
	\arrow["{\alpha_a}"', maps to, from=2-2, to=5-2]
	\arrow["{\alpha_x}", maps to, from=2-5, to=5-5]
	\arrow["Fg"', maps to, from=5-2, to=5-5]
	\arrow["Fg"', from=6-1, to=6-6]
\end{tikzcd}\]

It seems that $H_b$ can be replaced with any functor in $[\mathcal{C}^{op}, \textbf{Set}]$,
furthermore, the natural transformation is still determined by $\alpha_a(id_a)$
(and $\alpha_a(id_a)$ is determined by $\alpha$, trivial though, but it implies that there is a precisely corresponding).

\begin{theorem}[Yoneda Lemma]
  Show that the natural transformation between $H_a$ and any functor $F \in [\mathcal{C}^{op}, \textbf{Set}]$
  correspond precisely to the elements of $Fa$. In other words,
  \[
  [\mathcal{C}^{op}, \textbf{Set}](H_a, F) \cong Fa
  \]
\end{theorem}
\begin{proof}
  The arrow $f : [\mathcal{C}^{op}, \textbf{Set}](H_a, F) \rightarrow Fa$ is obvious
  by the Yoneda trick.

  \[
  \alpha \mapsto \alpha_a(id_a) : [\mathcal{C}^{op}, \textbf{Set}](H_a, F) \rightarrow Fa
  \]
  
  For any $x \in Fa$, as we see before, we may expect there is a natural transformation $\alpha$
  such that $\alpha_a (id_a) = x$, then the action on other objects is completely
  determined by $\alpha_a(id_a)$ as we see before. 

  \[
  x, g \mapsto (Fg)(x) : Fa \rightarrow [\mathcal{C}^{op}, \textbf{Set}](H_a, F)
  \]

  Note that $(F(id_a)) (x) = id_{Fa} (x) = x$.

  We need to show that they are inverse to each other.
  For any natural transformation $\alpha$:
  % https://q.uiver.app/#q=WzAsNixbMCwwLCJbXFxtYXRoY2Fse0N9XntvcH0sIFxcdGV4dGJme1NldH1dKEhfYSwgRikiXSxbMiwwLCJGYSJdLFs0LDAsIltcXG1hdGhjYWx7Q31ee29wfSwgXFx0ZXh0YmZ7U2V0fV0oSF9hLCBGKSJdLFswLDEsIlxcYWxwaGEiXSxbMiwxLCJcXGFscGhhX2EoaWRfYSkiXSxbNCwxLCIoRi0pKFxcYWxwaGFfYShpZF9hKSkiXSxbMCwxLCJmIl0sWzEsMiwiZl57LTF9Il0sWzMsNCwiLV9hKGlkX2EpIiwwLHsic3R5bGUiOnsidGFpbCI6eyJuYW1lIjoibWFwcyB0byJ9fX1dLFs0LDUsIihGLSkoLSkiLDAseyJzdHlsZSI6eyJ0YWlsIjp7Im5hbWUiOiJtYXBzIHRvIn19fV1d
\[\begin{tikzcd}
	{[\mathcal{C}^{op}, \textbf{Set}](H_a, F)} && Fa && {[\mathcal{C}^{op}, \textbf{Set}](H_a, F)} \\
	\alpha && {\alpha_a(id_a)} && {(F-)(\alpha_a(id_a))}
	\arrow["f", from=1-1, to=1-3]
	\arrow["{f^{-1}}", from=1-3, to=1-5]
	\arrow["{-_a(id_a)}", maps to, from=2-1, to=2-3]
	\arrow["{(F-)(-)}", maps to, from=2-3, to=2-5]
\end{tikzcd}\]

And we can see $(Fg)(\alpha_a(id_a)) = \alpha_x(g)$ for any $x \in \mathcal{C}$ and $g \in \mathcal{C}(x, a)$
by the same way as the proof which shows $H_\bullet$ is a full functor.

In another direction, we get:
% https://q.uiver.app/#q=WzAsNixbMCwwLCJGYSJdLFsyLDAsIltcXG1hdGhjYWx7Q31ee29wfSwgXFx0ZXh0YmZ7U2V0fV0oSF9hLCBGKSJdLFs0LDAsIkZhIl0sWzAsMSwieCJdLFsyLDEsIihGLSkoeCkiXSxbNCwxLCJGKGlkX2EpKHgpIl0sWzAsMSwiZl57LTF9Il0sWzEsMiwiZiJdLFszLDQsIihGLSkoLSkiLDAseyJzdHlsZSI6eyJ0YWlsIjp7Im5hbWUiOiJtYXBzIHRvIn19fV0sWzQsNSwiLV9hKGlkX2EpIiwwLHsic3R5bGUiOnsidGFpbCI6eyJuYW1lIjoibWFwcyB0byJ9fX1dXQ==
\[\begin{tikzcd}
	Fa && {[\mathcal{C}^{op}, \textbf{Set}](H_a, F)} && Fa \\
	x && {(F-)(x)} && {F(id_a)(x)}
	\arrow["{f^{-1}}", from=1-1, to=1-3]
	\arrow["f", from=1-3, to=1-5]
	\arrow["{(F-)(-)}", maps to, from=2-1, to=2-3]
	\arrow["{-_a(id_a)}", maps to, from=2-3, to=2-5]
\end{tikzcd}\]

It is obvious that $F(id_a)x = id_{Fa} x = x$.
\end{proof}

\end{document}