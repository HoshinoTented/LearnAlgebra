\documentclass[./main.tex]{subfiles}

\begin{document}

\section{Hausdorff spaces}

\begin{definition}
  A topological space $\mathcal{X}$ is called Hausdorff,
  if for each pair of distinct points $x, y \in \mathcal{X}$
  there are disjoint neighborhoods $x \in V$ and $y \in W$.
\end{definition}

\begin{theorem}
  Show that any converging sequence in a Hausdorff space has a unique limit.
\end{theorem}
\begin{proof}
  Suppose a sequence $x_n$ converage to $x$ and $y$,
  then there are disjoint neighborhoods $x \in V$ and $y \in W$.
  Since $x$ is the limit of $x_n$, so there is $i$ such that for any neighborhood
  of $x$, $x_j$ is in that neighborhood for any $j > i$.
  Similar to $y$, but $V$ and $W$ are disjoint, and $V$ contains infinite points
  of $x_n$ from some $i$, therefore $W$ contains finite points of $x_n$,
  which is unacceptible.
\end{proof}

\begin{theorem}
  Show that a topological space $\mathcal{X}$ is Hausdorff
  iff the diagonal
  \[
  \Delta = \set{(x, x) \in \mathcal{X} \times \mathcal{X}}
  \]
  is a closed set in the product space $\mathcal{X} \times \mathcal{X}$
\end{theorem}
\begin{proof}
  % Consider the union of $V \times W$ where $V$ and $W$ are disjoint neiborhoods
  % for some distinct $x, y \in \mathcal{X}$, then we can show that the union
  % is a open set in $\mathcal{X} \times \mathcal{X}$
  % and the complement is exactly $\Delta$.

  $(\Rightarrow)$ The set $S = \bigcup_{x, y \in \mathcal{X}, x \neq y} V_{(x, y)} \times W_{(x, y)}$
  is an open set in $\mathcal{X} \times \mathcal{X}$ where $V_{(x, y)}$ and $W_{(x, y)}$
  are disjoint neighborhoods of $x$ and $y$, respectively. It is easy to see that
  $(x, x) \notin S$ for any $x \in \mathcal{X}$, otherwise $(x, x)$ must belongs to
  some $V_{(x, x)} \times W_{(x, x)}$ while $V_{(x, x)}$ and $W_{(x, x)}$ are disjoint
  and $x \in V_{(x, x)} \cap W_{(x, x)}$. Also, there is no point that is not in $S$
  beside the point in $\Delta$. So $\Delta$ is the complement of $S$,
  and $S$ is open, so $\Delta$ is closed.
  
  $(\Leftarrow)$ For any distinct $x, y \in \mathcal{X}$, we have $(x, y) \in \Delta^C$.
  We know $\Delta^C$ is a union of products of open sets in $\mathcal{X}$,
  so we may suppose $x \in V$ and $y \in W$ where $V$ and $W$ are open sets in $\mathcal{X}$.
  Suppose $z \in V \cap W$, then $(z, z) \in \Delta^C$ and then $(z, z) \notin \Delta$,
  which is unacceptible.
\end{proof}

\begin{corollary}
  Any one-point set in a Hausdorff space is closed.
\end{corollary}
\begin{proof}
  For any Hausdorff space $\mathcal{X}$ and $p \in \mathcal{X}$,
  for any $y \in \mathcal{X}$ that $x \neq y$, we have a pair of disjoint neighborhood
  $V_y$ of $x$ and $W_y$ of $y$. Consider the union of these $W_y$,
  obviously it is an open set that contains every point in $\mathcal{X}$ beside $x$,
  so the one-point set $\{ x \}$ is closed.
\end{proof}

\begin{corollary}
  Any subspace of Hausdorff space is Hausdorff.
\end{corollary}
\begin{proof}
  Suppose $\mathcal{X}$ is Hausdorff and $S \subseteq \mathcal{X}$ a subspace.
  Consider $x, y \in S$ where $x \neq y$, we know there are disjoint neighborhood 
  $V$ and $W$ such that $V \cap W = \varnothing$, then $(V \cap S) \cap (W \cap S) = \varnothing$
  where $V \cap S$ and $W \cap S$ are open set in $S$ and $x \in V \cap S$ and $y \in W \cap S$.
\end{proof}

\begin{theorem}
  \label{theorem:1.3}
  Let $\mathcal{X}$ be a Hausdorff space and $K \subseteq \mathcal{X}$ be a compact subset.
  Then for any $y \notin K$, there are open sets $K \subseteq V$ and $y \in W$
  such that $V \cap W = \varnothing$.
\end{theorem}
\begin{proof}
  For any point $z \in K$, we have disjoint neighborhoods $y \in W_z$ and $z \in V_z$,
  consider the union of $V_z$, obviously it is an open cover on $K$, so there is
  a finite subcover $\{ V_{z_\alpha} \}$, then we consider the intersection of the corresponding
  $W_{z_\alpha}$, that is, $\bigcap_\alpha W_{z_\alpha}$, we can do this intersection
  cause the subcover is finite.
  Then let $V = \bigcup_\alpha V_{z_\alpha}$ and $W = \bigcap_\alpha W_{z_\alpha}$,
  obviously $V \cap W = \varnothing$.
\end{proof}

\begin{theorem}
  Any compact subset of Hausdorff is closed.
\end{theorem}
\begin{proof}
  Suppose $\mathcal{X}$ is Hausdorff and $K \subseteq \mathcal{X}$ a compact subset.
  For any $y \notin K$, we have $K \subseteq V_y$ and $y \in W_y$ such that $V_y \cap W_y = \varnothing$.
  Consider the union of these $W_y$, we can see that it is an open set and contains every point
  in $\mathcal{X} \setminus K$.
\end{proof}

\begin{theorem}
  Let $\mathcal{X}$ be a Hausdorff space and $K, L \subseteq \mathcal{X}$ be
  two compact subsets that $K \cap L = \varnothing$.
  Show that there are open sets $K \subseteq V$ and $L \subseteq W$ such that
  $V \cap W = \varnothing$
\end{theorem}
\begin{proof}
  For any $l \in L$, we have $K \subseteq V_l$ and $l \in W_l$ where
  $V_l \cap W_l = \varnothing$ by Theorem \ref{theorem:1.3}.
  Consider the union of these $W_l$, it is an open cover on $L$, therefore there is
  a finite subcover $\{ W_{l_\alpha} \}$. Then we can take the intersection of
  the corresponding $V_l$, that is, $V = \bigcap_\alpha V_{l_\alpha}$,
  and the union of the subcover $W = \bigcup_\alpha W_{l_\alpha}$.
  They are disjoint, otherwise there is $x \in V_{l_\alpha}$ and $x \in W_{l_\alpha}$ for some $\alpha$,
  which contradict to the property from Theorem \ref{theorem:1.3}.
\end{proof}

\end{document}