\documentclass[./main.tex]{subfiles}

\begin{document}

\section{Constructions}

\begin{definition}[Induced]
  Let $A$ a subset of a topological space $\mathcal{Y}$. Then
  \begin{itemize}
    \item all subsets $V \subseteq A$ such that $V = A \cap W$ for some open $W$ in $\mathcal{Y}$
  \end{itemize}
  forms a topology on $A$. This topology is called induced topology on $A$.
\end{definition}
\begin{proof}
  Obviously, $\varnothing = A \cap \varnothing$ and $A = A \cap A$.

  For any open set $V_0 = A \cap W_0$ and $V_1 = A \cap W_1$,
  then it is easy to see that $V_0 \cup V_1 = (A \cap W_0) \cup (A \cap W_1) = A \cup (W_0 \cap W_1)$,
  therefore for a collection $\{V_\alpha\}$ of open sets in $A$, $\bigcup_\alpha V_\alpha = A \cap (\bigcup_\alpha W_\alpha)$,
  and we know $\bigcup_\alpha W_\alpha$ is still a open set in $\mathcal{Y}$, so is $\bigcup_\alpha V_\alpha$.

  For any open set $V_0 = A \cap W_0$ and $V_1 = A \cap W_1$,
  we have $V_0 \cap V_1 = (A \cap W_0) \cap (A \cap W_1) = (A \cap W_0) \cap W_1 = A \cap (W_0 \cap W_1)$,
  so $V_0 \cap V_1$ is also a open set in $A$.
\end{proof}

\begin{definition}[Embedding]
  A map $f : \mathcal{X} \rightarrow \mathcal{Y}$ is called embedding if $f$ defines
  a homeomorphism from $\mathcal{X}$ to the subspace $f(\mathcal{X})$ in $\mathcal{Y}$.
\end{definition}

\begin{definition}[Product]
  Let $\mathcal{X}$ and $\mathcal{Y}$ are topology spaces, the product of
  $\mathcal{X}$ and $\mathcal{Y}$, $\mathcal{X} \times \mathcal{Y}$ is a topology
  space, which open set is a union of the product of open sets in $\mathcal{X}$ and $\mathcal{Y}$,
  that is, $\bigcup_i V_i \times W_i$ where $V_i$ is an open set of $\mathcal{X}$ 
  and $W_i$ is an open set of $\mathcal{Y}$ 
  (It won't work if the open set is \textit{just} a product of open sets from both space).
\end{definition}
\begin{proof}
  We need to show that $\mathcal{X} \times \mathcal{Y}$ is a topology space.
  \begin{enumerate}
    \item $\mathcal{X} \times \mathcal{Y}$ is an open set since both $\mathcal{X}$ and $\mathcal{Y}$ are open set,
          similarly, $\varnothing$ is an open set in $\mathcal{X} \times \mathcal{Y}$.
    \item Trivial.
    \item \[
          \bigcup_{a, b} (V_a \cap V_b) \times (W_a \cap W_b)
          \]
  \end{enumerate}
\end{proof}

\begin{exercise}
  Let $\mathscr{B}$ be a collection of open sets in a topological
  space $\mathcal{X}$. Show that $\mathscr{B}$ is base in $\mathcal{X}$
  iff for any point $x \in \mathcal{X}$ and any neighborhood $N$,
  there is $B \in \mathscr{B}$ such that $x \in B \subseteq N$.
\end{exercise}
\begin{proof}
  ~
  \begin{itemize}
    \item $(\Rightarrow)$ We know $N$ can be expressed as a union of
          some open sets in $\mathscr{B}$, that is, $N = B_0 \cup B_1 \cup \dots$.
          Then there is $B_i$ such that $x \in B_i$, and we know $B_i \subseteq N$,
          so $x \in B_i \subseteq N$.
    \item $(\Leftarrow)$ For any open set $A$, consider any point $x \in A$,
          we know $A$ is a neighborhood of $x$, so there is $B_x \in \mathscr{B}$
          such that $x \in B_x \subseteq A$. Then $\bigcup_{x \in A}B_x$
          is a subset of $A$, but note that $x \in B_x$, so 
          $\bigcup_{x \in A}B_x$ contains all points of $A$,
          so $\bigcup_{x \in A}B_x = A$.
  \end{itemize}
\end{proof}

\begin{theorem}
  Let $\mathscr{B}$ be a set of subsets in some set $\mathcal{X}$.
  Show that $\mathscr{B}$ is a base of some topology on $\mathcal{X}$ iff
  it satisfies the following conditions:
  \begin{enumerate}
    \item $\mathscr{B}$ coverse $\mathcal{X}$, that is, for any $x \in \mathcal{X}$,
          there is $B \in \mathscr{B}$ such that $x \in B$.
    \item For any $B_1, B_2\in \mathscr{B}$, $x \in B_1 \cap B_2$, then there exists
          a set $B \in \mathscr{B}$ such that $x \in B \subseteq B_1 \cap B_2$.
  \end{enumerate}
\end{theorem}
\begin{proof}
  $(\Rightarrow)$ Trivial by the previous exercise. \par
  $(\Leftarrow)$ Let $\mathscr{O}$ the set of all unions of sets of $\mathscr{B}$.
  We claim $\mathscr{O}$ is a topology on $\mathcal{X}$.
  First, $\emptyset \in \mathscr{O}$ since it is the result of union of no set.
  And $\mathcal{X} \in \mathscr{O}$ since $\mathscr{B}$ covers $\mathcal{X}$.
  Obviously, the union of any sets of $\mathscr{O}$ is in $\mathscr{O}$.

  Let $O_0, O_1 \in \mathscr{O}$, then $O_0 = \bigcup_i B_i$ and $O_1 = \bigcup_j B_j$.
  For any $x \in O_0 \cap O_1$, we have $x \in B_x \subseteq O_0 \cap O_1$ where $B_x \in \mathscr{O}$
  by $(2)$.
  Then $O_0 \cap O_1 = \bigcup_x = B_x$, therefore $O_0 \cap O_1 \in \mathscr{O}$
  cause it is a union of some set $B_x$.
\end{proof}

\begin{definition}[Prebase]
  Suppose $\mathscr{P}$ is a collection of subsets in $\mathcal{X}$ that covers the whole space.
  Show that the set of all finite intersections of sets in $\mathscr{P}$ is a base for \textit{some}
  topology on $\mathcal{X}$.
\end{definition}
\begin{proof}
  We denote all finite intersections of sets in $\mathscr{P}$ as $P$,
  then $\mathscr{P} \subseteq P$ since $\forall S \in P, S \cap S = S$.
  So $P$ covers $\mathcal{X}$. For any $B_0, B_1 \in P$ and $x \in B_0 \cap B_1$,
  it is obvious that $x \in B_0 \cap B_1 \subseteq B_0 \cap B_1 \in P$.
  So $\mathscr{P}$ is a base for some topology on $\mathcal{X}$ by the previous theorem.
\end{proof}

\begin{theorem}
  Let $\mathscr{P}$ be a prebase for the topology on $\mathcal{Y}$. Show that
  a map $f : \mathcal{X} \rightarrow \mathcal{Y}$ is continuous iff
  $\inv{f}(P)$ is open for any $P \in \mathscr{P}$.
\end{theorem}
\begin{proof}
  $(\Rightarrow)$ Obviously, every set in $\mathscr{P}$ is open. \par
  $(\Leftarrow)$ For any open set in $\mathcal{Y}$, it is a union of sets in
  $\mathscr{P}$, that is, $\bigcup_i P_i$. Then $\inv{f}(\bigcup_i P_i) = \bigcup_i \inv{f}(P_i)$
  which is a union of open sets, so it is also open.
\end{proof}

\begin{theorem}
  For any map $f : \mathcal{X} \rightarrow \mathcal{Y}$, the map
  $F : \mathcal{X} \rightarrow \mathcal{X} \times \mathcal{Y}$ given by
  $F(x) = (x, f(x))$. Show that $f(x)$ is continuous iff $F$
  is an embedding.
\end{theorem}
\begin{proof}
  $(\Rightarrow)$ We first show that $F : \mathcal{X} \rightarrow F(\mathcal{X})$ is continuous.
  For any open set $G$ in the induced topology on $F(\mathcal{X})$,
  it has form $F(\mathcal{X}) \cap (\bigcup_\alpha V_\alpha \times W_\alpha)$,
  where $(\bigcup_\alpha V_\alpha \times W_\alpha)$ is the open set in $\mathcal{X} \times \mathcal{Y}$.
  Clearly, $\inv{F}(G) = G.0 = \set{ x \mid \forall (x, y) \in G} = X \cap (\bigcup_\alpha V_\alpha)$,
  so it is an open set in $\mathcal{X}$.
  % The following is a stupid and massive proof of previous goal:
  % Then $\inv{F}(G) = \set{ x \mid \forall (x, y) \in G, x \in \inv{f}(y) }$.
  % We first consider $G.0 = \set{ x.0 | \forall x \in G }$, which is exactly 
  % $\mathcal{X} \cap (\bigcup_\alpha V_\alpha)$, similarly
  % $G.1 = f(\mathcal{X}) \cap (\bigcup_\alpha W_\alpha)$ and
  % $\inv{F}(G) = G.0 \cap \inv{f}(G.1)$.
  % Clearly $G.0$ is open in $\mathcal{X}$, and $\inv{f}(G.1) = \inv{f}(f(\mathcal{X}) \cap (\bigcup_\alpha W_\alpha))$ is in fact $\inv{f}(\bigcup_\alpha W_\alpha)$,
  % while $\bigcup_\alpha W_\alpha$ is open in $\mathcal{Y}$, so $\inv{f}(\bigcup_\alpha W_\alpha)$ is open in $\mathcal{X}$,
  % therefore $\inv{F}(G)$ is open in $\mathcal{X}$ cause it is an intersection of two open sets.
  It is easy to see $\inv{F} : F(\mathcal{X}) \rightarrow \mathcal{X}$ given by $\inv{F}(x, y) = x$ is
  the inverse of $F$, we need to show that it is continuous, or in other words,
  show that $F$ sends open to open.
  For any open set $V \subseteq \mathcal{X}$, $F(V) = F(\mathcal{X}) \cap (V \times \mathcal{Y})$,
  which is open in $F(\mathcal{X})$.
  So $F$ is an embedding. \par
  $(\Leftarrow)$ For any open set $G$ in $\mathcal{Y}$, we have $H = F(\mathcal{X}) \cap (\mathcal{X} \times G)$
  is open in the induced topology on $F(\mathcal{X})$, then $\inv{F}(H)$ is open in $\mathcal{X}$.
  We claim $\inv{f}(G) = \inv{F}(H)$. 
  \begin{itemize}
    \item $(\supseteq)$ For any $x \in \inv{F}(H)$,
    we know $f(x) \in f(\mathcal{X}) \cap G$ since $F(x) = (x, f(x)) \in H$, so $x \in \inv{f}(G)$
    since $f(x) \in G$. 
    \item $(\subseteq)$ For any $x \in \inv{f}(G)$, we know $f(x) \in G$,
    so $(x, f(x)) \in H$ since $x \in \mathcal{X} \cap \mathcal{X}$ and $f(x) \in f(\mathcal{X}) \cap G$,
    so $x \in \inv{F}(H)$. 
  \end{itemize}
  Therefore $f$ is continuous.
\end{proof}

\end{document}