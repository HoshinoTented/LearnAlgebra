\documentclass[./main.tex]{subfiles}

\begin{document}

\section{Compactness}

\begin{exercise}
  Let $\{V_\alpha\}$ be an open cover of a topological space $\mathcal{X}$.
  Show that $W \subseteq \mathcal{X}$ is open iff $W \cap V_\alpha$ is open for any
  $V_\alpha \in \{V_\alpha\}$.
\end{exercise}
\begin{proof}
  $(\Rightarrow)$ is trivial. \par
  $(\Leftarrow)$ For any point $x \in W$, we have $x \in V_x$ since $\{V_\alpha\}$ is open cover.
  Consider $\bigcup_{x \in W} (W \cap V_x)$, it is a union of open sets, and it is a subset of $W$,
  and it contains all points of $W$, so $W$ is open.
\end{proof}

\begin{theorem}
  Show that a space $\mathcal{X}$ is compact iff for any collection of closed sets
  $\{Q_\alpha\}$ in $\mathcal{X}$ such that
  \[
  \bigcap_\alpha Q_\alpha = \varnothing
  \]
  Then there is a finite subset of $\{Q_\alpha\}$ such that
  \[
  Q_0 \cap Q_1 \cap \dots \cap Q_n = \varnothing
  \]
\end{theorem}
\begin{proof}
  ~
  \begin{itemize}
    \item $(\Rightarrow)$ For any collection of closed set $\{Q_\alpha\}$ in $\mathcal{X}$
      such that $\bigcap_\alpha Q_\alpha = \varnothing$, we can see
      $\{ \mathcal{X} \setminus Q_\alpha \}$ is a collection of open set and
      $\bigcup_\alpha (\mathcal{X} \setminus Q_\alpha) = \mathcal{X} \setminus (\bigcap_\alpha Q_\alpha) = \mathcal{X} \setminus \varnothing = \mathcal{X}$,
      therefore the collection of complements $\{\mathcal{X} \setminus Q_\alpha \}$ is an open cover of $\mathcal{X}$.
      So there is a finite subset of $\{ \mathcal{X} \setminus Q_\alpha \}$ that
      also open covers $\mathcal{X}$. Then it is also a finite subset of $\{ Q_\alpha \}$
      since $\mathcal{X} = \bigcup_i (\mathcal{X} \setminus Q_i) = \mathcal{X} \setminus (\bigcap_i Q_i)$
      and $Q_i \subseteq \mathcal{X}$ implies $\bigcap_i Q_i = \varnothing$.
    \item $(\Leftarrow)$ For any open cover $\{ V_\alpha \}$ of $\mathcal{X}$,
      consider the collection of complements $\{ \mathcal{X} \setminus V_\alpha \}$,
      we have $\bigcap_\alpha (\mathcal{X} \setminus V_\alpha) = \mathcal{X} \setminus (\bigcup_\alpha V_\alpha) = \varnothing$.
      So there is a finite subset $\{ \mathcal{X} \setminus V_i \}$ such that
      $\varnothing = \bigcap_i (\mathcal{X} \setminus V_i) = \mathcal{X} \setminus (\bigcup_i V_i)$,
      therefore $\bigcup_i V_i = \mathcal{X}$, the space $\mathcal{X}$ is compact.
  \end{itemize}
\end{proof}

\begin{theorem}
  Let $Q_0 \supseteq Q_1 \supseteq \dots$ be a nested sequence of closed
  nonempty sets in a compact space $\mathcal{K}$. Show that there is a point
  $q \in \mathcal{K}$ such that $\forall i, q \in Q_i$.
\end{theorem}
\begin{proof}
  If there is no such point, we know $\bigcap_i Q_i = \varnothing$,
  which means there is a finite subsequence such that $\bigcap_j Q_j = \varnothing$.
  Since the sequence $Q_i$ is a nested sequence of nonempty sets, so $\bigcap_j Q_j$
  must equal to some "smallest" $Q_j$, but that means this $Q_j$ is empty set,
  which contradicts to the assumption.
\end{proof}

\begin{theorem}
  Let $f : \mathcal{X} \rightarrow \mathcal{Y}$ be a continuous mapping
  between topological spaces and $\mathcal{K}$ is a compact subset in $\mathcal{X}$.
  Show that $\mathcal{Q} = f(\mathcal{K})$ is also compact in $\mathcal{Y}$.
  That is, continuous mapping preserve compactness.
\end{theorem}
\begin{proof}
  For any open cover $\{V_\alpha\}$ of $f(\mathcal{K})$, then the inverse images of 
  $\{V_\alpha\}$ cover
  are also open since $f$ is continous, and also an open cover of $\mathcal{K}$
  since they cover $f(\mathcal{K})$. So there is a finite subset of open cover
  such that covers $\mathcal{K}$, then map those cover by $f$, we get
  a finite subset of open cover of $\{V_\alpha\}$
\end{proof}

\begin{theorem}
  Any closed set in a compact space is also compact.
\end{theorem}
\begin{proof}
  This proof comes from textbook.\par
  Suppose $\mathcal{X}$ a compact space and $\mathcal{Q}$ a closed set in it.
  Consider any open cover $\{ V_\alpha \}$ of $\mathcal{Q}$ and
  the complement $\mathcal{C} = \mathcal{X} \setminus \mathcal{Q}$.
  Obviously, $\mathcal{C}$ is open, and $\{ \mathcal{C} \} \cup \{ V_\alpha \}$
  is an open cover of $\mathcal{X}$, therefore there is a finite open cover on
  $\mathcal{X}$, that open cover may or may not contains $\mathcal{W}$, but we 
  can always add $\mathcal{W}$ to it, and it is still a finite subcover.
  Since the finite subcover $\{ \mathcal{W}, V_{\alpha_0}, \dots, V_{\alpha_n} \}$
  covers $\mathcal{X}$, then it also covers $\mathcal{Q}$,
  and we can see that $\mathcal{W}$ contributes nothing for $\mathcal{Q}$,
  so it is safe to remove it, and $\{ V_{\alpha_i} \}$ is still a finite
  open cover on $\mathcal{Q}$.

  Furthermore, the proposition can be \textit{iff}, since
  the whole space is a closed set. If any closed set in that space is
  compact, then the whole space is also compact.
\end{proof}

\begin{definition}
  Let $\{ V_\alpha \}$ and $\{ W_\beta \}$ be two covers of a topological space
  $\mathcal{X}$. We say $\{ V_\alpha \}$ is inscried in $\{ W_\beta \}$
  if for any $\alpha$, there is $\beta$ such that $V_\alpha \subseteq W_\beta$
\end{definition}

\begin{theorem}
  A space $\mathcal{X}$ is compact iff for any cover $\{ V_\alpha \}$ of $\mathcal{X}$,
  there is a finite cover such that inscried in $\{ V_\alpha \}$.
\end{theorem}
\begin{proof}
  $(\Rightarrow)$ For any cover on $\mathcal{X}$, there is a finite subcover on $\mathcal{X}$,
  and the finite subcover is an inscried in itself.\par
  $(\Leftarrow)$ For any finite subcover $\{ W_\beta \}$ on $\mathcal{X}$ that is inscried in $\{ V_\alpha \}$,
  since for any $W_\beta$ there is a $V_\alpha$ such that $W_\beta \subseteq V_\alpha$,
  we may collect these $V_\alpha$ and they form a cover on $\mathcal{X}$.
\end{proof}

\begin{theorem}
  Suppose $\mathcal{X}$ and $\mathcal{Y}$ are compact topological spaces,
  show that the product $\mathcal{X} \times \mathcal{Y}$ is also compact.
\end{theorem}
\begin{proof}

\end{proof}

\begin{theorem}
  Suppose that a product space $\mathcal{X} \times \mathcal{Y}$ is nonempty and compact.
  Show that $\mathcal{X}$ and $\mathcal{Y}$ are compact.
\end{theorem}
\begin{proof}
  For any open cover $\iset{\mathcal{I}}{V}$ on $\mathcal{X}$,
  consider the open cover $\{ V_\alpha \times \mathcal{Y} \}_{\alpha \in \mathcal{I}}$
  on $\mathcal{X} \times \mathcal{Y}$,
  there is a finite subcover
  $\{ V_\alpha \times \mathcal{Y} \}_{\alpha \in \mathcal{J}}$.
  For any $x \in \mathcal{X}$, take $y \in \mathcal{Y}$ (it is possible cause $\mathcal{X} \times \mathcal{Y}$ is nonempty),
  we have $(x, y) \in V_\alpha \times \mathcal{Y}$ for some $\alpha$,
  so $\iset{\mathcal{J}}{V}$ is a open cover on $\mathcal{X}$.
\end{proof}

\begin{definition}
  A topological space $\mathcal{X}$ is called sequentially compact if any
  point sequence in $\mathcal{X}$ has a converging subsequence.
\end{definition}

\begin{theorem}
  A metric space $\mathcal{M}$ is comapct implies it is sequentially compact.
\end{theorem}
\begin{proof}
  Recall that if a sequence $x_n$ converage to some point $p$, then
  for any $\epsilon$, $B(p, \epsilon)$ contains infinite points in $x_n$.
  % We can always choose a smaller $\epsilon$ and it will contains at least one
  % point in $x_n$, so there are infinite points of $x_n$ in $B(p, \epsilon)$.

  If any subsequence in $x_n$ is not converging, then for any point $p \in \mathcal{M}$,
  there is $\epsilon$ such that $B(p, \epsilon)$ contains finite points in $x_n$.
  Consider the collection of $B(p, \epsilon)$ for every $p \in \mathcal{M}$
  is a cover on $\mathcal{M}$, then we have a finite subcover on $\mathcal{M}$,
  but then the subcover contains finite points in $\mathcal{M}$
  while $x_n$ is infinite.
\end{proof}

\begin{theorem}
  Show that the product of two sequentially compact spaces is sequentially compact.
\end{theorem}
\begin{proof}
  TODO
\end{proof}

\begin{definition}
  A sequence $x_n$ of points in a metric space is called Cauchy if for any
  $\epsilon > 0$ there is $n$ such that $|x_i - x_j| < \epsilon$ for all $i, j > n$.
\end{definition}

\begin{theorem}
  Any converging sequence in a metric space is Cauchy.
\end{theorem}
\begin{proof}
  For any converging sequence, the distance between points becomes smaller and smaller,
  so for any $\epsilon$, there is $n$ such that $|x_i - x_j| < \epsilon$ for all $i, j > n$.
\end{proof}

\begin{definition}
  A metric space $\mathcal{M}$ is called complete if any Cauchy sequence in $\mathcal{M}$ converage to a point in $\mathcal{M}$.
\end{definition}

\begin{theorem}
  Show that any compact metric space $\mathcal{M}$ is complete.
\end{theorem}
\begin{proof}
  For any Cauchy sequence $x_n$, suppose it is not converage, that is, for any
  $p \in \mathcal{M}$, there is $\epsilon$ such that $B(p, \epsilon)$ contains
  finite points of $x_n$. Consider the cover $\{ B(p, \epsilon) \}$ for all $p \in \mathcal{M}$
  and corresponding $\epsilon$ on $\mathcal{M}$, we know there is a finite subcover on
  $\mathcal{M}$ since $\mathcal{M}$ is compact. Then these finite subcover contains
  all points in $x_n$ cause it covers $\mathcal{M}$, and it contains finite points in $x_n$
  cause each $B(p, \epsilon)$ contains finite points in $x_n$, therefore the sequence $x_n$ 
  is finite, which is unacceptible.
\end{proof}

\begin{definition}
  Let $\mathcal{M}$ be a metric space. A subset $A \subseteq \mathcal{M}$
  is called $\epsilon$-net of $\mathcal{M}$ if for any $p \in \mathcal{M}$,
  there is $a \in A$ such that $|p - a|_\mathcal{M} < \epsilon$ 
  (or equivalently, $p \in B(a, \epsilon)$).
\end{definition}

\begin{theorem}
  Let $\mathcal{M}$ be a sequentially compact metric space, then for any $\epsilon > 0$,
  there is a finite $\epsilon$-net of $\mathcal{M}$.
\end{theorem}
\begin{proof}
  This proof comes from textbook.\par
  We may trying to construct an $\epsilon$-net of $\mathcal{M}$.
  We pick a point in $\mathcal{M}$ randomly, say $x_0 \in \mathcal{M}$,
  then we pick another point $x_1 \in \mathcal{M}$ such that $x_1 \notin B(x_0, \epsilon)$,
  and then we pick $x_2 \in \mathcal{M}$ such that $x_2 \notin B(x_0, \epsilon)$ and $x_2 \notin B(x_1, \epsilon)$,
  for any $i$, we pick $x_i \in \mathcal{M}$ such that $x_i \notin B(x_j, \epsilon)$ for any $j < i$.
  If at some point, we can't pick any $x_i$ that satisfies the requirement, then
  $\{ x_0, x_1, \dots, x_{i - 1} \}$ is an $\epsilon$-net of $\mathcal{M}$.
  If this procedure cannot stop, then we get a sequence $x_i$ where their
  distance are always greater than $\epsilon$. Since $\mathcal{M}$ is sequentially compact,
  so there is a converging subsequence, however, the distance of points in the subsequence 
  can not below $\epsilon$, so it can't be converging.
\end{proof}

\end{document}