\documentclass[../main.tex]{subfiles}

\setcounter{section}{3}

\begin{document}

\begin{exercise}
  Suppose $\V$ is a finite vector space, Show that the only two ideal of
  $\LT(\V)$ is $\0$ and $\LT(\V)$.
  A subspace $\mathcal{E}$ of $\LT(\V)$ is called an ideal, 
  if $TE \in \mathcal{E}$ and $ET \in \mathcal{E}$
  for any $T \in \LT(\V)$ and $E \in \mathcal{E}$.
\end{exercise}
\begin{proof}
  We will use the concept Matrix.
  Suppose $\lambda_0v_0 + \cdots + \lambda_nv_v$ the basis of $V$.
  We want to construct $T_i$ that $T(\lambda_0v_0 + \cdots + \lambda_nv_n) = \lambda_iv_i$ for all $0 \le i < n$,
  which is a matrix with all zero but $1$ at $i, i$.

  For any matrix, we can always select a non-zero value at $a, b$ and place it at $i, b$,
  this can be done by left multiply a matrix with $1$ at $i, a$
  (this produce a vector at line $i$ with values from line $a$),
  then right multiply a matrix with $1$ at $i, b$
  (this produce a vector at column $b$ with values from line $i$).

  Also, we can always select a non-zero value at $a, b$ and place it at $a, i$,
  this can be done by right multiply a matrix with $1$ at $b, i$,
  then left multiply a matrix with $1$ at $a, i$.

  By combining these two operations, we ca select a non-zero value at $a, b$ and place it at $i, i$.
  Now, consider any non-zero $E \in \mathcal{E}$, we can construct a
  matrix with non-zero value at $i, i$ for every $0 \le i < \dim V$. These matrix
  are in $\mathcal{E}$ since $\mathcal{E}$ is an ideal, then we can multiply an apropriate scalar
  to them so that they are matrices with $1$ at $i, i$. By adds up these matrices, we get $I$,
  we know $I \in \mathcal{E}$ since $\mathcal{E}$ is a vector space, and now all $T \in \LT(V)$
  is also in $\mathcal{E}$ since $\mathcal{E}$ is an ideal, then $\mathcal{E} = \LT(V)$.

  The only exception is $\mathcal{E} = \0$, in this case we can't pick any non-zero element.

  Another solution, hope this one is more simple.

  Suppose $\mathcal{E}$ an ideal of $\LT(V)$ and non-zero, non-surjective $E \in \mathcal{E}$.
  Let $\join{v}{k - 1}$ a basis of $\nullv E$ and $v_k, \cdots, v_{k + n}$ such that $Tv_{k + i}$
  is a basis of $\rangev E$, then we have $n \neq 0$ and $k \neq 0$.

  Define $A$ a linear transformation which maps $v_i$ to $v_{k + i}$ for $0 \le i < \min\{k, n\}$ and maps others to $0$,
  then $\dim \rangev EA = \min\{k, n\}$.

  Expand the basis $w_i = Ev_{k + i}$ of $\rangev E$ to a basis of $V$, say $\join{w}{m - 1}$, define $B$
  maps $Ev_{k + i}$ to $w_{\min\{k, n\} + i}$, we always have enough $w_{\min\{k, n\} + i}$ since
  $m - 1 = \dim V = \dim \nullv E + \dim \rangev E$ while $\min{k, n} \le \dim \nullv E$,
  then $\dim \rangev BE = \rangev E$ since we just re-map the $\rangev E$.
  
  Now consider $S = EA + BE$, we have $Sv_i = EAv_i = Ev_{k + i} = w_i \in \rangev E$ for all $0 \le i < \min\{k, n\}$
  and $Sv_{\min\{k, n\} + i} = BEv_{\min\{k, n\} + i} = w_{\min\{k, n\} + i} \in \rangev BE$ for all $0 \le i < \dim \rangev E$.
  We can see $\rangev EA \cap \rangev BE = \0$ and $\dim \rangev (EA + BE) = \rangev E + \min\{k, n\}$,
  where $k = \nullv E$ and $n = \rangev E$, the range of $EA + BE$ gets larger and $EA + BE \in \mathcal{E}$
  since $EA, BE \in \mathcal{E}$, if $k > n$ (this is the only case that $EA + BE$ is not surjective),
  then we continue this process with $E = EA + BE$, the procedure will finally terminate
  since $\LT(V)$ is finite (cause $V$ is finite).

  Now we show that any $\mathcal{E}$ with non-zero, non-surjective $E \in \mathcal{E}$ implies a surjective (thus injective and invertible) $T \in \mathcal{E}$.

  For any ideal with an invertible element $E \in \mathcal{E}$, we have $\inv{E}E = I \in \mathcal{E}$,
  which causes $\mathcal{E} = \LT(V)$ since $IT = T$ for all $T \in \LT(V)$.

  Therefore, only $\0$ and $\LT(V)$ are ideals of $\LT(V)$.
\end{proof}

\end{document}