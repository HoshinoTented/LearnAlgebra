\documentclass[../main.tex]{subfiles}

\setcounter{section}{3}

\begin{document}

\begin{exercise}
  Suppose $\V$ is a finite vector space, Show that the only two ideal of
  $\LT(\V)$ is $\0$ and $\LT(\V)$.
  A subspace $\mathcal{E}$ of $\LT(\V)$ is called an ideal, 
  if $TE \in \mathcal{E}$ and $ET \in \mathcal{E}$ 
  for any $T \in \LT(\V)$ and $E \in \mathcal{E}$.
\end{exercise}
\begin{proof}
  We will use the concept Matrix.
  Suppose $\lambda_0v_0 + \cdots + \lambda_nv_v$ the basis of $V$.
  We want to construct $T_i$ that $T(\lambda_0v_0 + \cdots + \lambda_nv_n) = \lambda_iv_i$ for all $0 \le i < n$,
  which is a matrix with all zero but $1$ at $i, i$.

  For any matrix, we can always select a non-zero value at $a, b$ and place it at $i, b$,
  this can be done by left multiply a matrix with $1$ at $i, a$
  (this produce a vector at line $i$ with values from line $a$),
  then right multiply a matrix with $1$ at $i, b$
  (this produce a vector at column $b$ with values from line $i$).

  Also, we can always select a non-zero value at $a, b$ and place it at $a, i$,
  this can be done by right multiply a matrix with $1$ at $b, i$,
  then left multiply a matrix with $1$ at $a, i$.

  By combining these two operations, we ca select a non-zero value at $a, b$ and place it at $i, i$.
  Now, consider any non-zero $E \in \mathcal{E}$, we can construct a
  matrix with non-zero value at $i, i$ for every $0 \le i < \dim V$. These matrix
  are in $\mathcal{E}$ since $\mathcal{E}$ is an ideal, then we can multiply an apropriate scalar
  to them so that they are matrices with $1$ at $i, i$. By adds up these matrices, we get $I$,
  we know $I \in \mathcal{E}$ since $\mathcal{E}$ is a vector space, and now all $T \in \LT(V)$
  is also in $\mathcal{E}$ since $\mathcal{E}$ is an ideal, then $\mathcal{E} = \LT(V)$.

  The only exception is $\mathcal{E} = \0$, in this case we can't pick any non-zero element.
\end{proof}

\end{document}