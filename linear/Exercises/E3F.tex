\documentclass[../main.tex]{subfiles}

\setcounter{section}{3}

\begin{document}

\begin{exercise}
  Explain why a linear functional is either surjective or $0$.
\end{exercise}
\begin{proof}
  Cause $\dim F = 1$.
\end{proof}

\setcounter{exercise}{5}
\begin{exercise}
  Let $\varphi, \beta \in V^\prime$, show that $\nullv \varphi \subseteq \nullv \beta \iff \exists c \in F, \beta = c \varphi$.
\end{exercise}
\begin{proof}
  ~
  \begin{itemize}
    \item $(\Rightarrow)$ For any $v \notin \nullv \beta$, we have $\beta(v) = \beta(v)\inv{(\varphi(v))}\varphi(v)$
          we claim that $\beta = \beta(v)\inv{(\varphi(v))}\varphi$. We may denote $\beta(v)\inv{(\varphi(v))}$ by $c$.
          For any $v, w \notin \nullv \beta$, we have $\beta(v) = a\varphi(v)$  and $\beta(w) = b\varphi(w)$,
          we want to show that $a = b$, which can be proven by:
          \begin{align*}
            a & = b \\
            \frac{\beta(v)}{\varphi(v)} & = \frac{\beta(w)}{\varphi(w)} \\
            \beta(v)\varphi(w) & = \beta(w)\varphi(v) \\
            \beta(\varphi(w) v) & = \beta(\varphi(v) w) \\
          \end{align*}
          which is equivalent to $\varphi(w) v - \varphi(v) w \in \nullv \beta$, then:
          \begin{align*}
             & \varphi(\varphi(w) v - \varphi(v) w) \\
            =& \varphi(\varphi(w) v) - \varphi(\varphi(v) w) \\
            =& \varphi(w) \varphi(v) - \varphi(v) \varphi(w) \\
            =& 0
          \end{align*}
          therefore $\varphi(w) v - \varphi(v) w \in \nullv \varphi \subseteq \nullv \beta$,
          thus $a = b$.

          The case $v \in \nullv \beta$ is trivial.
    \item $(\Leftarrow)$ For any $v \in \nullv \varphi$, $\beta(v) = c \varphi(v) = 0$,
          therefore $v \in \nullv \beta$, thus $\nullv \varphi \subseteq \nullv \beta$.
  \end{itemize}
\end{proof}

\begin{exercise}
  Let $\join{V}{m - 1}$ are vector spaces, show that $\Times{V^\prime}{m - 1}$ and
  $(\Times{V}{m - 1})^\prime$ are isomorphic.
\end{exercise}
\begin{proof}
  Define $\psi(\varphi) = v_0 \mapsto \varphi(v_0, 0, \cdots), \cdots, v_{m - 1} \mapsto \varphi(\cdots, 0, v_{m - 1})$
  and \\
  $\inv{\psi}(\join{\varphi}{m - 1}) = (\join{v}{m - 1}) \mapsto \varphi_0(v_0) + \cdots + \varphi_{m - 1}(v_{m - 1})$.

  For any $\alpha, \beta \in (\Times{V}{m - 1})^\prime$ and $\lambda \in F$,
  we have
  \begin{align*}
     & \psi(\alpha + \beta)_i \\
    =& v_i \mapsto (\alpha + \beta)(\cdots, v_i, \cdots) \\
    =& v_i \mapsto \alpha(\cdots, v_i, \cdots) + \beta(\cdots, v_i, \cdots) \\
    =& (v_i \mapsto \alpha(\cdots, v_i, \cdots)) + (v_i \mapsto \beta(\cdots, v_i, \cdots)) \\
    =& \psi(\alpha)_i + \psi(\beta)_i
  \end{align*}
  and $(\lambda \psi(\alpha))_i = \lambda (v_i \mapsto \alpha(v_i)) = v_i \mapsto \lambda \alpha(v_i) = \psi(\lambda \alpha)_i$
  Therefore $\psi$ is a linear map.

  For any $\alpha, \beta \in \Times{V^\prime}{m - 1}$ and $\lambda \in F$,
  we have:
  \begin{align*}
   & \inv{\psi}(\alpha + \beta) \\
  =& (\join{v}{m - 1}) \mapsto (\alpha + \beta)(v_0) + \cdots + (\alpha + \beta)(v_{m - 1}) \\
  =& (\join{v}{m - 1}) \mapsto \alpha(v_0) + \beta(v_0) + \cdots \\
  =& ((\join{v}{m - 1}) \mapsto \alpha(v_0) + \cdots) + ((\join{v}{m - 1}) \mapsto \beta(v_0) + \cdots) \\
  =& \inv{\psi}(\alpha) + \inv{\psi}(\beta)
  \end{align*}
  and
  \begin{align*}
     & \lambda\inv{\psi}(\alpha) \\
    =& \lambda ((\join{v}{m - 1}) \mapsto \alpha(v_0) + \cdots) \\
    =& (\join{v}{m - 1}) \mapsto \lambda (\alpha(v_0) + \cdots) \\
    =& (\join{v}{m - 1}) \mapsto (\lambda \alpha(v_0)) + \cdots \\
    =& \inv{\psi}(\lambda \alpha)
  \end{align*}
  thus $\inv{\psi}$ is a linear map.
  
  We will show that $\inv{\psi}$ is the inverse of $\psi$ then $\psi$ is an isomorphism.
  For any $\varphi \in (\Times{V}{m - 1})^\prime$,
  \begin{align*}
    & \inv{\psi}(\psi(\varphi))\\
   =& \join{v}{m - 1} \mapsto \psi(\varphi)_0(v_0) + \cdots \\
   =& \join{v}{m - 1} \mapsto \varphi(v_0, 0, \cdots) + \cdots + \varphi(\cdots, 0, v_{m - 1}) \\
   =& \join{v}{m - 1} \mapsto \varphi(\join{v}{m - 1}) \\
   =& \varphi
  \end{align*}
  and for any $\varphi \in \Times{V^\prime}{m - 1}$,
  \begin{align*}
     & \psi(\inv{\psi}(\varphi)) \\
    =& v_0 \mapsto \inv{\psi}(\varphi)(v_0, 0, \cdots), \cdots \\
    =& v_0 \mapsto \varphi_0(v_0) , \cdots \\
    =& \join{\varphi}{m - 1} \\
    =& \varphi
  \end{align*}
\end{proof}

\setcounter{exercise}{15}
\begin{exercise}
  Let $W$ a finite vector space, $T \in \LT(V, W)$, show that
  \[
  T^\prime = 0 \iff T = 0
  \]
\end{exercise}
\begin{proof}
  ~
  \begin{itemize}
    \item $(\Rightarrow)$ Suppose $T \neq 0$, then we can always find $\varphi \in \LT(W, F)$
          which $\varphi(\rangev T) \neq 0$, then $\varphi \circ T \neq 0$.
    \item $(\Leftarrow)$ Trivial.
  \end{itemize}
\end{proof}

\begin{exercise}
  Let $V, W$ are finite vector spaces, $T \in \LT(V, W)$. Show that 
  $T$ is invertible $\iff$ $T^\prime$ is invertible.
\end{exercise}
\begin{proof}
  Since $T$ is invertible, then $T$ is injective, therefore $T^\prime$ is surjective.
  Similarly, $T^\prime$ is injective since $T$ is surjective.
  Therefore $T^\prime$ is invertible.
\end{proof}

\begin{exercise}
  Let $V, W$ are finite vector spaces, show that the mapping \\ 
  $\varphi(T) = T^\prime$ is an isomorphism between $\LT(V, W)$ and $\LT(W^\prime, V^\prime)$.
\end{exercise}
\begin{proof}
  Since $V$ and $W$ are finite, we only need to show that $\varphi$ is injective or surjective.
  We will show that $\varphi$ is injective.

  For any $\varphi(T) = T^\prime \in \LT(W^\prime, V^\prime)$, we know $T = 0 \iff T^\prime = 0$,
  therefore $\nullv \varphi = \0$, thus $\varphi$ is injective.

  I was wonder if I can prove this by $\varphi(S)(\mathrm{id}) = \varphi(T)(\mathrm{id}) \implies S = T$.
  This may work if the codomain restriction doesn't lose information, i.e. only restrict to a superset of its range, therefore the restriction is one-to-one.
\end{proof}

\setcounter{exercise}{20}
\begin{exercise}
  Let $V$ finite and $U, W \subseteq V$ are subspaces.
  \begin{enumerate}
    \item Show that $W^0 \subseteq U^0 \iff U \subseteq W$
    \item Show that $W^0 = U^0 \iff U = W$
  \end{enumerate}
\end{exercise}
\begin{proof}
  The second statement can be easy proved by the first one.
  \begin{itemize}
    \item $(\Rightarrow)$ We can always find a $f \in \LT(W, F)$ such that $\nullv f = W$,
          then $f(U) = \0$ since $f \in W^0 \subseteq U^0$, therefore $U \subseteq \nullv f = W$.
    \item $(\Leftarrow)$ For any $\varphi \in W^0$, we know $W \subseteq \nullv \varphi$,
          then $U \subseteq W \subseteq \nullv \varphi$, therefore $\varphi in U^0$,
          thus $W^0 \subseteq U^0$.
  \end{itemize}
\end{proof}

\begin{exercise}
  Let $V$ finite and $U, W \subseteq V$ are subspaces. Show that:
  \begin{itemize}
    \item $(U + W)^0 = U^0 \cap W^0$
    \item $(U \cap W)^0 = U^0 + W^0$
  \end{itemize}
\end{exercise}
\begin{proof}
  ~
  \begin{itemize}
    \item For any $\varphi \in (U + W)^0$ we have $U + W \subseteq \nullv \varphi$, 
          then $U \subseteq U + W \subseteq \nullv \varphi$ and $W \subseteq U + W$,
          therefore $\varphi \in U^0 \cap W^0$.
          
          For any $\varphi \in U^0 \cap W^0$, we have $U \subseteq \nullv \varphi$ and $W \subseteq \nullv \varphi$.
          For any $u + w \in U + W$, we have $\varphi(u + w) = \varphi(u) + \varphi(w) = 0 + 0 = 0$,
          therefore $U + W \subseteq \nullv \varphi$, thus $\varphi \in (U + W)^0$.
    \item For any $su + tw \in U^0 + W^0$, for any $v \in U \cap W$, we have $su(v) + tw(v) = s0 + t0$
          since $v \in U$ and $v \in W$. Therefore we have an injective map (also linear, this map just produce what it receive)
          from $U^0 + W^0$ to $(U \cap W)^0$.
          We have:
          \begin{align*}
             & \dim (U^0 + W^0) \\
            =& \dim U^0 + \dim W^0  - \dim (U^0 \cap W^0) \\
            =& \dim V - \dim U + \dim V - \dim W - \dim (U + W)^0 \\
            =& \dim V - \dim U + \dim V - \dim W - (\dim V - \dim (U + W)) \\
            =& \dim V - \dim U - \dim W + (\dim U + \dim W - \dim (U \cap W)) \\
            =& \dim V - \dim (U \cap W) \\
            =& \dim (U \cap W)^0
          \end{align*}
          therefore $(U \cap W)^0 = U^0 + W^0$.
  \end{itemize}
\end{proof}

\begin{exercise}
  Let $V$ finite and $\join{\varphi}{m - 1} \in V^\prime$. Show that the following sets are equal to each others:
  \begin{itemize}
    \item $\spanv(\join{\varphi}{m - 1})$
    \item $((\nullv \varphi_0) \cap \cdots \cap (\nullv \varphi_{m - 1}))^0$
    \item $\set{\varphi \in V^\prime}{(\nullv \varphi_0) \cap \cdots \cap (\nullv \varphi_{m - 1}) \subseteq \nullv \varphi}$
  \end{itemize}
\end{exercise}
\begin{proof}
  ~
  \begin{itemize}
    \item $((\nullv \varphi_0) \cap \cdots \cap (\nullv \varphi_{m - 1}))^0 = (\nullv \varphi_0)^0 + \cdots + (\nullv \varphi_{m - 1})^0$,
          then $\spanv(\varphi_i) \subseteq (\nullv \varphi_i)^0$ therefore $\spanv(\join{\varphi}{m - 1}) \subseteq ((\nullv \varphi_0) \cap \cdots \cap (\nullv \varphi_{m - 1}))^0$.
          
          For any $\varphi \in \spanv(\join{\varphi}{m - 1})$, we have $\varphi(v) = \varphi_0(v) + \cdots + \varphi_{m - 1}(v) = 0 + \cdots + 0 = 0$
          for any $v \in (\nullv \varphi_0) \cap \cdots \cap (\nullv \varphi_{m - 1})$, therefore $\varphi \in ((\nullv \varphi_0) \cap \cdots \cap (\nullv \varphi_{m - 1}))^0$.
    \item Last two sets are definitional equal.
  \end{itemize}
\end{proof}

\setcounter{exercise}{23}
\begin{exercise}
  Let $V$ finite and $\join{v}{m - 1} \in V$. \\
  Define $\Gamma(\varphi) = (\varphi(v_0), \cdots \varphi(v_{m - 1})) : V^\prime \rightarrow F^m$,
  show that:
  \begin{itemize}
    \item $\join{v}{m - 1}$ spans $V$ $\iff$ $\Gamma$ is injective.
    \item $\join{v}{m - 1}$ is linear independent $\iff$ $\Gamma$ is surjective.
  \end{itemize}
\end{exercise}
\begin{proof}
  ~
  \begin{itemize}
    \item $(\Rightarrow)$ Suppose $\Gamma(\alpha) = \Gamma(\beta)$ , then for all $v \in V$ can be factorized into $\joinp[+]{\lambda}{v}{m - 1}$,
          then $\alpha(v) = \alpha(\joinp[+]{\lambda}{v}{m - 1}) = \beta(\joinp[+]{\lambda}{v}{m - 1}) = \beta(v)$
          since $\Gamma(\alpha) = \Gamma(\beta)$ and $\alpha$ and $\beta$ are linear map,
          thus $\alpha = \beta$.

          $(\Rightarrow)$ We first make $\join{v}{m - 1}$ linear independent, say $\join{v}{k - 1}$,
          then for any $w \in V$ such that $\join{v}{k - 1}, w$ is linear independent,
          then we have its dual basis $\join{\varphi}{k - 1}, \psi$.
          Consider $\Gamma(\psi)$, by definition, we know $\Gamma(\psi) = (\psi(v_0), \cdots) = (0, \cdots)$
          then $\psi = 0$ since $\Gamma$ is injective, which contradicts our assumption.
          Therefore $\join{v}{k - 1}$ spans $V$.
    \item $(\Rightarrow)$ Consider the dual basis of $\join{v}{m - 1}$, then $\Gamma$ is surjective
          since we have the standard basis of $F^m$.

          $(\Leftarrow)$ $\Gamma$ is surjective implies we have $\join{\varphi}{m - 1}$
          such that $\Gamma(\varphi_i) = (\cdots, 1, \cdots)$,
          which means $\join{v}{m - 1}$ is linear independent.
  \end{itemize}
\end{proof}

\begin{exercise}
  Let $V$ finite and $\join{\varphi}{m - 1} \in V^\prime$. \\
  Define $\Gamma(v) = (\varphi_0(v), \cdots, \varphi_{m - 1}(v)) : V \rightarrow F^m$.
  Show that
  \begin{itemize}
    \item $\join{\varphi}{m - 1}$ spans $V^\prime$ $\iff$ $\Gamma$ is injective
    \item $\join{\varphi}{m - 1}$ is linear independent $\iff$ $\Gamma$ is surjective
  \end{itemize}
\end{exercise}
\begin{proof}
  ~
  \begin{itemize}
    \item $(\Rightarrow)$ Suppose $\Gamma(v) = \Gamma(w)$, then $\varphi_i(v) = \varphi_i(w)$,
          which means $\varphi_i(v - w) = 0$ for all $i$. If $v - w \neq 0$, then $((\nullv \varphi_0) \cap \cdots \cap (\nullv \varphi_{m - 1}))^0 \neq \0$,
          thus $\join{\varphi}{m - 1}$ doesn't span $V^\prime$.

          $(\Leftarrow)$ $(\nullv \varphi_0) \cap \cdots \cap (\nullv \varphi_{m - 1}) = \0$ since $\Gamma$ is injective.
          therefore $\spanv(\join{\varphi}{m - 1}) = ((\nullv \varphi_0) \cap \cdots \cap (\nullv \varphi_{m - 1}))^0 = (\0)^0 = V^\prime$
    \item $(\Rightarrow)$ We may treat $\Gamma$ as the following matrix:
          \[
          \begin{bmatrix}
            \quad \varphi_0 \quad \\
            \vdots \\
            \varphi_{m - 1}
          \end{bmatrix}
          \]
          which line rank is $m$ since $\join{\varphi}{m - 1}$ is linear independent,
          therefore its column rank is $m$, thus $\dim \rangev \Gamma = m = \dim F^m$, then $\Gamma$ is surjective.

          $(\Leftarrow)$ It seems the proof of $(\Rightarrow)$ also works here.
  \end{itemize}
\end{proof}

\begin{exercise}
  Let $V$ finite, and $\Omega \subseteq V^\prime$ a subspace. Show that
  \[
  \Omega = \set{v \in V}{\varphi(v) = 0 \quad \forall \varphi \in \Omega}^0
  \]
\end{exercise}
\begin{proof}
  This construction looks like an inverse of $-^0$.

  We may rewrite the equation to $\Omega = (\displaystyle \bigcap_{\varphi \in \Omega} \nullv \varphi)^0$,
  then $\Omega = \spanv(\varphi) \forall \varphi \in \Omega$, which is trivial.
\end{proof}

\setcounter{exercise}{27}
\begin{exercise}
  Let $V$ finite and $\join{\varphi}{m - 1}$ is linear independent. Show that
  \[
  \dim ((\nullv \varphi_0) \cap \cdots \cap (\nullv \varphi_{m - 1})) = \dim V - m
  \]
\end{exercise}
\begin{proof}
  \begin{align*}
    m & = \dim \spanv(\join{\varphi}{m - 1}) \\
      & = \dim ((\nullv \varphi_0) \cap \cdots \cap (\nullv \varphi_{m - 1}))^0 \\
      & = \dim V - \dim ((\nullv \varphi_0) \cap \cdots \cap (\nullv \varphi_{m - 1}))
  \end{align*}
\end{proof}

\setcounter{exercise}{29}
\begin{exercise}
  Let $V$ finite and $\join{\varphi}{m - 1}$ a basis of $V^\prime$. Show that
  there is a basis of $V$ which dual basis is $\join{\varphi}{m - 1}$.
\end{exercise}
\begin{proof}
  Since $\join{\varphi}{m - 1}$ spans $V^\prime$ and linear independent, we know $\Gamma$
  is both injective and surjective. Consider $\join{v}{m - 1}$ such that $\Gamma(v_i) = (\cdots, 0, 1, 0, \cdots)$.
  We claim $\join{v}{m - 1}$ is a basis of $V$ and which dual basis if $\join{\varphi}{m - 1}$.

  The second part is trivial by the way construct them.
  For the first part, $\join{v}{m - 1}$ is linear independent since $(\cdots, 0, 1, 0 \cdots)$ is
  linear independent, and $\join{v}{m - 1}$ spans $V$ since $\dim V = \dim V^\prime = m$.
\end{proof}

\begin{exercise}
  Let $U \subseteq V$ a subspace and $i(u) = u : U \rightarrow V$. Then $i^\prime \in \LT(V^\prime, U^\prime)$,
  show that:
  \begin{enumerate}
    \item $\nullv i^\prime = U^0$
    \item $\rangev i^\prime = U^\prime$ if $V$ is finite
    \item $\tilde{i^\prime}$ is an isomorphism between $V^\prime / U^0$ and $U^\prime$ if $V$ is finite
  \end{enumerate}
\end{exercise}
\begin{proof}
  ~
  \begin{itemize}
    \item For any $\varphi \in \nullv i^\prime$, $\varphi \circ i = 0$, therefore $\rangev i = U \subseteq \nullv \varphi$,
          thus $\varphi \in U^0$.

          For any $\varphi \in U^0$, $\varphi \circ i = 0$ since $\rangev i = U \subseteq \nullv \varphi$.
    \item Suppose $V$ is finite, then $i^\prime$ is surjective since $i^\prime$ is injective, therefore $\rangev i^\prime = U^\prime$.
    \item $\tilde{i^\prime}(\varphi + U^0) = i^\prime(\varphi)$ is surjective since $i^\prime$ is surjective.
          Then $\dim (V^\prime / U^0) = \dim V^\prime - \dim U^0 = \dim V - (\dim V - \dim U) = \dim U = \dim U^\prime$,
          therefore $\tilde{i^\prime}$ is an isomorphism.
  \end{itemize}
\end{proof}

\begin{exercise}
  We denote $V^{\prime\prime}$ as the \textbf{double dual space of} $V$, defined by $V^{\prime\prime} = (V^\prime)^\prime$.
  Define $\Lambda(v)(\varphi) = \varphi(v) : V \rightarrow V^{\prime\prime}$

  Show that:
  \begin{enumerate}
    \item $\Lambda \in \LT(V, V^{\prime\prime})$
    \item Let $V \in \LT(V)$, then $T^{\prime\prime} \circ \Lambda = \Lambda \circ T$ where $T^{\prime\prime} = (T^\prime)^\prime$.
    \item $\Lambda$ is an isomorphism if $V$ is finite.
  \end{enumerate}
\end{exercise}
\begin{proof}
  ~
  \begin{itemize}
    \item For any $v, w \in V$ and $\lambda \in F$, we have $(\Lambda(v) + \Lambda(w))(\varphi) = \Lambda(v)(\varphi) + \Lambda(w)(\varphi) = \varphi(v) + \varphi(w) = \varphi(v + w) = \Lambda(v + w)(\varphi)$
          and $(\lambda \Lambda(v))(\varphi) = \lambda (\Lambda(v)(\varphi)) = \lambda (\varphi(v)) = \varphi(\lambda v) = \Lambda(\lambda v)(\varphi)$.
    \item For any $v \in V$, 
          \begin{align*}
             & (T^{\prime\prime} \circ \Lambda)(v)(\varphi) \\
            =& (T^{\prime\prime} (\Lambda (v)))(\varphi) \\
            =& ((\Lambda(v)) \circ T^\prime)(\varphi) \\
            =& \Lambda(v)(T^\prime(\varphi)) \\
            =& \Lambda(v)(\varphi \circ T) \\ 
            =& (\varphi \circ T)(v) \\
            =& \varphi (T(v)) \\
            =& \Lambda(T(v))(\varphi) \\
            =& (\Lambda \circ T)(v)(\varphi)
          \end{align*}
    \item Suppose $\Lambda(v) = \Lambda(w)$, that is, $\Lambda(v)(\varphi) = \varphi(v) = \varphi(w) = \Lambda(w)(\varphi)$ for all $\varphi \in \V^\prime$.
          Let $\join{\varphi}{m - 1}$ the dual basis of some basis of $V$,
          then $v = \varphi_0(v)v_0 + \cdots + \varphi_{m - 1}(v)v_{m - 1} = \varphi_0(w)v_0 + \cdots + \varphi_{m - 1}(w)v_{m - 1} = w$.
          Therefore $\Lambda$ is injective, thus surjective and isomorphism since $\dim V = \dim V^{\prime\prime}$.
  \end{itemize}
\end{proof}

\begin{exercise}
  Let $U \subseteq V$ a subspace and $\pi : V \rightarrow V/U$ the quotient map, then $\pi^\prime \in \LT((V/U)^\prime, V^\prime)$.
  \begin{enumerate}
    \item Show that $\pi^\prime$ is injective.
    \item Show that $\rangev \pi^\prime = U^0$.
    \item Conclude that $\pi^\prime$ is an isomorphism between $(V/U)^\prime$ and $U^0$.
  \end{enumerate}
\end{exercise}
\begin{proof}
  ~
  \begin{itemize}
    \item $\pi$ is surjective, therefore $\pi^\prime$ is injective. The statement is true
          even $V$ or $V/U$ may be infinite, cause the proof about surjective-implies-epimorphism
          doesn't require that the codomain is finite but epimorphism-implies-surjective does.

          We may prove those theorem again, but with weaker assumption.
          For any $\pi^\prime(\varphi) = \pi^\prime(\psi)$, we have $\varphi \circ \pi = \psi \circ \pi$.
          For any $v + U \in V/U$, there is $v \in V$ such that $\pi(v) = v + U$ since $\pi$ is surjective.
          Therefore $\varphi(\pi(v)) = \psi(\pi(v))$ for all $\pi(v) = v + U \in V/U$,thus $\varphi = \psi$.

          Therefore $\pi^\prime$ is injective.
    \item $\rangev \pi^\prime = (\nullv \pi)^0 = U^0$.
    \item Trivial.
  \end{itemize}
\end{proof}

\end{document}