\documentclass[../main.tex]{subfiles}

\setcounter{section}{5}

\begin{document}

\begin{exercise}
  Prove or disprove: $T \in \LT(V)$ and $\mathcal{M}(T^2)$ is upper-triangular for some basis of $V$,
  then $\mathcal{M}(T)$ is upper-triangular for some basis of $V$ (not necessary the same as the $\mathcal{M}(T^2)$ one).
\end{exercise}
\begin{proof}
  WoBuHui.
\end{proof}

\begin{exercise}
  Let $A, B$ are upper-triangular matrices with same size,
  the diagonal of $A$ is $\join{\alpha}{n - 1}$
  and the diagonal of $B$ is $\join{\beta}{n - 1}$.
  Show that
  \begin{itemize}
    \item $A + B$ is upper-triangular and the diagonal is $\alpha_0 + \beta_0, \cdots, \alpha_{n - 1} + \beta_{n - 1}$.
    \item $AB$ is upper-triangular and the diagonal is $\alpha_0\beta_0, \cdots, \alpha_{n - 1}\beta_{n - 1}$.
  \end{itemize}
\end{exercise}
\begin{proof}
  ~
  \begin{itemize}
    \item Trivial.
    \item Take the standard basis of $F^n$, we have $Bv_i \in \spanv(\join{v}{i})$
          and then $A(Bv_i) \in \spanv(\join{v}{i})$ since both $A$ and $B$
          are upper-triangular, thus $AB$ is upper-triangular.
          For the diagonal, we know $AB_{i, i} = A_{i, -} B_{-, i}$,
          however, components before $i$-th of $A_{i, -}$ are $0$ and componetns since $i-th$ of $B_{-, i}$ are $0$,
          therefore $AB_{i, i} = A_{i, i} B_{i, i} = \alpha_i \beta_i$.
  \end{itemize}
\end{proof}

\begin{exercise}
  Let $T \in \LT(V)$ invertible, and $\mathcal{M}(T)$ with respect to the basis $\join{v}{n - 1}$ of $V$
  is upper-triangular, while the diagonal is $\join{\lambda}{n - 1}$.
  Show that $\mathcal{M}(\inv{T})$ with respect to that basis is also upper-triangular,
  and the diagonal is $\frac{1}{\lambda_0}, \cdots, \frac{1}{\lambda_{n - 1}}$.
\end{exercise}
\begin{proof}
  For any $i = 1, \cdots, n$, $\spanv(\join{v}{i - 1})$ is invariant under $T$, thus it is invariant under $\inv{T}$
  since $\inv{T}$ is the inverse of $T$.

  For the diagonal, $T\inv{T} = I$, which diagonal is $1, \cdots, 1$, which
  is equal to $\lambda_0 \beta_0, \cdots, \lambda_{n - 1} \beta_{n - 1}$
  where $\beta_i$ is the diagonal of $\inv{T}$.
  Thus $\beta_i = \frac{1}{\lambda_i}$.
\end{proof}

\begin{exercise}
  Give an example that $T$ an invertible operator, where the diagonal of $\mathcal{M}(T)$ is all $0$.
\end{exercise}
\begin{proof}
  \[
  \begin{bmatrix}
    0 & 1 \\
    1 & 0 \\
  \end{bmatrix}
  \]
\end{proof}

\begin{exercise}
  Give an example that $T$ an singular operator, where the diagonal of $\mathcal{M}(T)$ is all non-zero.
\end{exercise}
\begin{proof}
  \[
  \begin{bmatrix}
    1 & 1 \\
    1 & 1 \\
  \end{bmatrix}
  \]
\end{proof}

\begin{exercise}
  Let $F = C$ and $V$ finite, and $T \in \LT(V)$. Show that $k = 1, \cdots, \dim V$,
  then there is a $k$-dimension subspace of $V$ that is invariant under $T$.
\end{exercise}
\begin{proof}
  If $F = C$, then $\mathcal{M}(T)$ is upper-triangular for some basis of $V$.
  Thus $\spanv(\join{v}{k - 1})$ is invariant under $T$ where $v_i$ is such basis.
\end{proof}

\begin{exercise}
  Let $V$ finite and $T \in \LT(V)$ and $v \in V$. Show that:
  \begin{itemize}
    \item There is a unique monic polynomial $p_v$ with minimal degree such that $p_v(T)v = 0$
    \item Show that the minimal polynomial of $T$ is polynomial multiple of $p_v$.
  \end{itemize}
\end{exercise}
\begin{proof}
  ~
  \begin{itemize}
    \item $p(T)v = 0$, therefore we only need to show the uniqueness.
          Let $s, t$ a monic polynomial with minimal degree such that $s(T)v = t(T)v = 0$,
          then $(s - t)(T)v = 0$, therefore $s = t$, otherwise there is a polynomial $s - t$ with lower degree such that $(s - t)(T)v = 0$.
    \item We divide $p$ by $p_v$, then $p = s p_v + r$ where $s, r \in \mathcal{P}(F)$
          and $\deg r < \deg p_v$. Therefore $r = 0$, otherwise $r$ is a lower polynomial
          such that $r(T)v = 0$, which contradict the property of $p_v$.
          Thus $p = sp_v$.
  \end{itemize}
\end{proof}

\begin{exercise}
  Let $V$ finite and $T \in \LT(V)$, and non-zero $v \in V$ such that $T^2 + 2Tv + 2v = 0$.
  Show that
  \begin{itemize}
    \item If $F = R$, then $\mathcal{M}(T)$ is \textbf{NOT} upper-triangular for all basis of $V$.
    \item If $F = C$, then the diagonal of upper-triangular $\mathcal{M}(T)$ contains $-1 + i$ and $-1 - i$.
  \end{itemize}
\end{exercise}
\begin{proof}
  ~
  \begin{itemize}
    \item Note that $p_v(z) = z^2 + 2z + 2$ is a minimal polynomial of $Tv$,
          it is minimal since $p_v$ has no zero, therefore cannot have lower degree.

          Then the minimal polynomial $p$ of $T$ is a polynomial multiple of $p_v$,
          thus $p$ is \textbf{NOT} in form of $(z - \lambda_0) \cdots (z - \lambda_{n - 1})$
          since $p_v$ has no zero, thus there is no upper-triangular matrix for $T$ for any basis of $V$.
    \item $- 1 + i$ and $- 1 - i$ are two zeros of $p_v$, thus are zeros of $p$, therefore are
          in the diagonal.
  \end{itemize}
\end{proof}

\begin{exercise}
  Let $B$ square matrix with complex elements. Show that there is a square matrix $A$
  with complex elements such that $\inv{A}BA$ is a upper-triangular matrix.
\end{exercise}
\begin{proof}
  We can find an operator $T$ such that its matrix is $B$ with respect to the standard basis.
  Then we can find a basis such that $\mathcal{M}(T)$ with respect to such basis is upper-triangular
  since $B$ is complex.
  Then \\ $A = \mathcal{M}(I, \text{standard basis}, \text{upper-trianguler basis})$,
  and $\inv{A}BA$ is upper-triangular, this is the change-of-basis formula.
\end{proof}

\begin{exercise}
  Let $T \in \LT(V)$ and $\join{v}{n - 1}$ a basis of $V$, show that the following statements
  are equivalent:
  \begin{itemize}
    \item the matrix of $T$ with respect to $\join{v}{n - 1}$ is lower-triangular.
    \item For any $k = 1, \cdots, n$, $\spanv(v_{k - 1}, \cdots, v_{n - 1})$ is invariant under $T$.
    \item For any $k = 1, \cdots, n$, $Tv_{k - 1} \in \spanv(v_{k - 1}, \cdots, v_{n - 1})$.
  \end{itemize}
\end{exercise}
\begin{proof}
  The proof is similar to the upper-triangular one.

  \begin{itemize}
    \item $(1) \Rightarrow (2)$ For any $i \le j$,
          $Tv_{j - 1} \in \spanv(v_{j - 1}, \cdots, v_{n - 1}) \subseteq \spanv(v_{i - 1}, \cdots, v_{n - 1})$,
          thus $\spanv(v_{k - 1}, \cdots, v_{n - 1})$ is invariant under $T$.
    \item $(2) \Rightarrow (3)$ Tirival.
    \item $(3) \Rightarrow (1)$ Basically the definition.
  \end{itemize}
\end{proof}

\begin{exercise}
  Let $F = C$ and $V$ finite. Show that $T \in \LT(V)$, then $\mathcal{M}(T)$ is
  lower-triangular with respect to some basis of $V$.
\end{exercise}
\begin{proof}
  Consider the dual map $T^\prime$, we know there is a basis of $V^\prime$
  such that $\mathcal{M}(T^\prime)$ is upper-triangular, then $\mathcal{M}(T^\prime) = \mathcal{M}(T)^T$
  which means $\mathcal{M}(T)^T$ is a upper-triangular, thus $(\mathcal{M}(T)^T)^T = \mathcal{M}(T)$
  is lower-triangular.
\end{proof}

\begin{exercise}
  Let $V$ finite and the matrix of $T \in \LT(V)$ is upper-triangular with respect to some basis of $V$,
  and $U \subseteq V$ is invariant under $T$.
  Show that
  \begin{itemize}
    \item The matrix of $T\big|_U$ is upper-triangular with respect to some basis of $U$.
    \item The matrix of $T/U$ is upper-triangular with respect to some basis of $V/U$.
  \end{itemize}
\end{exercise}
\begin{proof}
  ~
  \begin{itemize}
    \item Since $\mathcal{M}(T)$ is upper-triangular, then the minimal polynomial of $T$
          is in form of $p(z) = (z - \lambda_0) \cdots (z - \lambda_{n - 1})$.
          Then $p(T\big|_U) = 0$, thus $p$ is polynomial multiple of the minimal polynomial $q$ of $T\big|_U$.
          therefore $q$ is also in form of $(z - \lambda_0) \cdots (z - \lambda_{k - 1})$.
          Thus there is a basis of $U$ such that the matrix of $T\big|_U$ is upper-triangular.
    \item Let $q$ the minimal polynomial of $T/U$, and $p$ the minimal polynomial of $T$, 
          then $p$ is polynomial multiple of $q$ (see Exercise 5.25 in E5B).
          Then follow the same step as last proof.
  \end{itemize}
\end{proof}

\begin{exercise}
  Let $V$ finite, $T \in \LT(V)$, $U \subseteq V$ invariant under $T$,
  $\mathcal{M}(T\big|_U)$ is upper-triangular for some basis of $U$,
  $\mathcal{M}(T/U)$ is upper-triangular for some basis of $V/U$.
  Show that $\mathcal{M}(T)$ is upper-triangular for some basis of $V$.
\end{exercise}
\begin{proof}
  We will use the conclusion of Exercise 5.25 in E5B: 
  \[
  st = (\text{the minimal polynomial of } T\big|_U) \times (\text{the minimal polynomial of } T/U)
  \]
  is a polynomial multiple of the minimal polynomial $p$ of $T$.
  Thus $st$ is in form of $(z - \lambda_0) \cdots (z - \lambda_{n - 1})$
  since both $\mathcal{M}(T\big|_U)$ and $\mathcal{M}(T/U)$ are upper-triangular
  for some basis, therefore $p$ is also in form of $(z - \lambda_0) \cdots (z - \lambda_{k - 1})$,
  hence $\mathcal{M}(T)$ is upper-triangular for some basis of $V$.
\end{proof}

\end{document}