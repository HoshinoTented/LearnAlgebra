\documentclass[../main.tex]{subfiles}

\usepackage{graphicx}
\graphicspath{ {../resources/} }

\setcounter{section}{5}

\begin{document}

\setcounter{theorem}{38}
\begin{theorem}\label{T5.39}
  Let $T \in \LT(V)$ and $\join{v}{n - 1}$ a basis of $V$, then the following
  statements are equivalent to each others.
  \begin{itemize}
    \item $\mathcal{M}(T)$ about $\join{v}{n - 1}$ is upper-triangular matrix
    \item For any $k = 1, \cdots, n$, $\spanv(\join{v}{k - 1})$ is invariant under $T$.
    \item For any $k = 1, \cdots, n$, $Tv_{k - 1} \in \spanv(\join{v}{k - 1})$.
  \end{itemize}
\end{theorem}
\begin{proof}
  ~
  \begin{itemize}
    \item $(1) \Rightarrow (2)$ Induction on $k$. In breif, first $k$ columns
          are in $\spanv(\join{v}{k - 1})$, therefore $\spanv(\join{v}{k - 1})$
          is invariant under $T$.
    \item $(2) \Rightarrow (3)$ Trivial, $v_{k - 1} \in \spanv(\join{v}{k - 1})$ which is invariant under $T$.
    \item $(3) \Rightarrow (1)$ Basically the definition of upper-triangular matrix, $Tv_{k - 1} \in \spanv(\join{v}{k - 1})$
          means the $k$-th column of $\mathcal{M}(T)$ consists of first $k$ number (the coefficients of $Tv_{k - 1}$) and $0$s.
  \end{itemize}
\end{proof}

\begin{theorem}
  \label{a}
  Let $T \in \LT(V)$ and $\join{v}{n - 1}$ a basis of $V$, such that $\mathcal{M}(T)$
  is upper-triangular matrix, and $\join{\lambda}{n - 1}$ are
  the numbers of its diagonal. Show that
  \[
  (T - \lambda_0 I) \cdots (T - \lambda_{n - 1}I) = 0
  \]
\end{theorem}
\begin{proof}
  All numbers since $i$ of $(T - \lambda_i I)v$ are $0$ if numbers after $i$ of $v$ are $0$.
  Thus $(T - \lambda_0 I) \cdots (T - \lambda_{n - 1}I)$ makes $n - 1$-th number $0$,
  then $n - 2$-th number and so on.
\end{proof}

\begin{theorem}
  \label{T5.41}
  Let $T \in \LT(V)$ and $\mathcal{M}(T)$ about some basis of $V$ is upper-triangular matrix.
  Show that the eigenvalues of $T$ are the numbers in the diagonal.
\end{theorem}
\begin{proof}
  Let $\join{v}{n - 1}$ a basis of $V$ and $\mathcal{M}(T)$ about this basis is upper-triangular matrix.
  For $\lambda_i$ where $i = 0, \cdots, n - 1$, we will show that $T - \lambda_i I$ is
  not invertible.

  We will see first $i$ columns of $T - \lambda_i I$ is linear dependent,
  since they have at most $i - 1$ non-zero numbers while the list they form has length $i$.

  Another part of proof follows the book.
  Let $q(z) = (z - \lambda_0) \cdots (z - \lambda_{n - 1})$, then $q(T) = 0$
  by \ref{a}, thus $q$ is polynomial multiple of the minimal polynomial of $T$,
  thus any zero of the minimal polynomial of $T$ is also a zero of $q$, which means
  it belongs to the list $\join{\lambda}{n - 1}$.
\end{proof}

\setcounter{theorem}{43}
\begin{theorem}
  Let $V$ finite and $T \in \LT(V)$. Show that $\mathcal{M}(T)$ is upper-triangular matrix
  about some basis of $V$ $\iff$ the minimal polynomial of $T$ is in form of $(z - \lambda_0)\cdots(z - \lambda_{n - 1})$
  where $\lambda_i \in F$.
\end{theorem}
\begin{proof}
  This proof comes from the book.

  The $(\Rightarrow)$ part follows \cref{T5.41}, $(z - \lambda_0) \cdots (z - \lambda_{m - 1})$
  ($\join{\lambda}{m - 1}$ are the numbers in the diagonal) is polynomial multiple of 
  the minimal polynomial of $T$, then the minimal polynomial of $T$
  must in a similar form.

  For $(\Leftarrow)$, we will induction on $n$.
  \begin{itemize}
    \item Base($n = 0$), the minimal polynomial of $T$ is in form $(z - \lambda_0)$,
          thus $T = \lambda_0 I$.
    \item Ind($n = n + 1$), the minimal polynomial of $T$ is in form $p(z) = (z - \lambda_0) \cdots (z - \lambda_{n + 1 - 1})$.
          Consider $T - \lambda_nI$, there is non-zero $v \in V$ such that $(T - \lambda_nI)v = 0$
          since $\lambda_n$ is a zero of $p$ therefore an eigenvalue of $T$.
          We define $U = \rangev (T - \lambda_n I)$, consider $q(z) = (z - \lambda_0)\cdots(z - \lambda_{n - 1})$
          and then $q(T\big|_U) = 0$, recall that $U = \rangev (T - \lambda_n I)$, therefore for any $v \in U$, there is $u$ such that $(T - \lambda_nI)u = v$.
          Thus $q(T\big|_U)u = q(T)(T - \lambda_n I)v = p(T)v = 0$ where $u = (T - \lambda_n I) v$ and $u, v \in U$.
          Thus the matrix of $T\big|_U$ is upper-triangular.

          Then we consider $\join{u}{k - 1}$ a basis of $U$, we will expand $\join{u}{k - 1}$
          to a basis of $V$, say $\join{u}{k - 1}\join{v}{m - 1}$.
          Then for any $v_i$ where $i $, we have $Tv_i = Tv_i - \lambda_nv_i + \lambda_n v_i = (T - \lambda_n I)v_i + \lambda_n v_i$,
          where $(T - \lambda_n) v_i \in U$ (recall the definition of $U$),
          thus $Tv_i \in \spanv(\join{u}{k - 1}, \join{v}{i})$, then by \ref{T5.39},
          we know $\mathcal{M}(T)$ is an upper-triangular matrix about
          the basis $\join{u}{k - 1}, \join{v}{m - 1}$.

          One may confused that the "length" of $u_i$ is greater than the size of $\mathcal{M}(T\big|_U) = \dim U$
          (thus it won't be a square matrix but tall and thin),
          however, these two things are unrelated, a matrix only represents how to combine
          the basis, and doesn't care what the basis looks like.
  \end{itemize}
\end{proof}

\end{document}