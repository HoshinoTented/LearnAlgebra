\documentclass[14pt]{extarticle}
\usepackage[T1]{fontenc}
\usepackage[margin=1in]{geometry}
\usepackage{amsthm,amsmath}
\usepackage{hyperref}

\newtheorem{theorem}{Theorem}[section]
\newtheorem{lemma}{Lemma}[section]
\newtheorem*{example}{Example}
\newtheorem{definition}{Definition}[section]
\setcounter{section}{10}

\DeclareMathOperator{\Ker}{Ker}
\DeclareMathOperator{\Aut}{Aut}

\newcommand{\inv}[1]{#1^{-1}}
\newcommand{\normalin}{\triangleleft}
\newcommand{\1}{\{e\}}

\begin{document}

\begin{definition}[Homomorphism]
  For any group $G$ and $\overline{G}$, a 
  \textbf{homomorphism}
  is a mapping from $G$ to $\overline{G}$ that preserves structure.
  That is, $\forall a \ b \in G, \phi(ab) = \phi(a)\phi(b)$.
\end{definition}

\begin{definition}[Kernel of Homomorphism]
  The \textbf{kernel} of a homomorphism $\phi : G \rightarrow \overline{G}$
  is the set $\{ x \in G \ | \ \phi(x) = e \}$. 
  The kernel of $\phi$ is denoted by $\Ker \phi$
\end{definition}

The following lemmas assumes $\phi : G \rightarrow \overline{G}$ is a homomorphism.

\begin{lemma}[Homomorphism Preserves identity]
  \label{lemma10.1}
  $\phi(e_G) = e_{\overline{G}}$.
\end{lemma}
\begin{proof}
  $\phi(g) = \phi(eg) = \phi(e) \phi(g)$, by cancellation, $\phi(e) = e_{\overline{G}}$.
\end{proof}

\begin{lemma}[Homomorphism Preserves Inverse]
  $\forall g \in G, \phi(\inv{g}) = \inv{\phi(g)}$.
\end{lemma}
\begin{proof}
  $e_{\overline{G}} = \phi(e_G) = \phi(\inv{g}g) = \phi(\inv{g}) \phi(g)$, then $\inv{\phi(g)} = \phi(\inv{g})$.
\end{proof}

\begin{lemma}[Homomorphism Preserves Power]
  $\forall g \in G, n \in Z, \phi(g^n) = \phi(g)^n$.
\end{lemma}
\begin{proof}
  Induction on $n$.
  \begin{itemize}
    \item Base: $\phi(g^0) = \phi(g)^0$ by Lemma \ref{lemma10.1}.
    \item Negative Direction: 
      \begin{align*}
         & \phi(g^{-(n + 1)}) \\
        =& \phi(g^{-n - 1}) \\
        =& \phi(g^{-n} \inv{g}) \\
        =& \phi(g^{-n})\phi(\inv{g}) \\
        =& \phi(g^{-n})\inv{\phi(g)} \\
        =& \phi(g)^{-n}\inv{\phi(g)} && \text{(By induction hypothesis)} \\
        =& \phi(g)^{-(n + 1)}
      \end{align*}
    \item Positive Direction: Similarly to the negative direction.
  \end{itemize}
\end{proof}

\begin{lemma}[Image of Homomorphism is Subgroup]
  $\phi(G) = \{ \phi(g) \ | \ g \in G \}$ is subgroup of $\overline{G}$.
\end{lemma}
\begin{proof}
  By three-steps:
  \begin{itemize}
    \item $\phi(e) \in \phi(G)$
    \item For any $\phi(a) \ \phi(b) \in \phi(G)$, $\phi(a)\phi(b) = \phi(ab) \in \phi(G)$.
    \item For ang $\phi(a) \in \phi(G)$, $\inv{\phi(a)} = \phi(\inv{a}) \in \phi(G)$.
  \end{itemize}
\end{proof}

\begin{lemma}[Homomorphism on Order]
  For any $g \in G$, if $|g|$ is finite, $|\phi(g)|$ divides $|g|$;
  If $|G|$ is finite, $|\phi(g)|$ divides $|g|$ and $|\phi(G)|$.
\end{lemma}
\begin{proof}
  Let $|g| = n$, $\phi(g)^n = \phi(g^n) \rightarrow \phi(g)^n = e$, then $|\phi(g)|$ divides $n$.

  Since $|G|$ is finite, so is $|g|$, and we proved $|\phi(g)|$ divides $|g|$.
  Since $\phi(G)$ is a subgroup of $\overline{G}$ and $\phi(g) \in \phi(G)$, $|\phi(g)|$ divides $|\phi(G)|$.
\end{proof}

\begin{lemma}[Kernel is Subgroup]
  $\Ker \phi$ is a subgroup of $G$.
\end{lemma}
\begin{proof}
  By three-steps:

  \begin{itemize}
    \item $\phi(e) = e$ so $e \in \Ker \phi$.
    \item For any $a \ b \in \Ker \phi$, $\phi(ab) = \phi(a)\phi(b) = e$ so $ab \in \Ker \phi$.
    \item For any $a \in \Ker \phi$, $\phi(\inv{a}) = \inv{\phi(a)} = \inv{e} = e$ so $\inv{a} \in \Ker \phi$.
  \end{itemize}
\end{proof}

\begin{lemma}
  \label{lemma:10.7}
  For any $a \ b \in G$, $\phi(a) = \phi(b) \iff a \Ker \phi = b \Ker \phi$.
\end{lemma}
\begin{proof}
  $(\Longrightarrow)$ $\inv{\phi(a)}\phi(b) = \phi(\inv{a})\phi(b) = \phi(\inv{a}b) = e \rightarrow \inv{a}b \in \Ker \phi$ 
  then $a \Ker \phi = b \Ker \phi$.

  $(\Longleftarrow)$ $a \Ker \phi = b \Ker \phi$ 
  imples $\inv{a}b \in \Ker \phi$ 
  imples $\phi(\inv{a}b) = e$
  imlpes $\phi(\inv{a}) \phi(b) = e$
  imples $\inv{\phi({a})} \phi(b) = e$
  imples $\phi(a) = \phi(b)$
\end{proof}

\begin{lemma}[Inverse Image of Homomorphism]
  For any $g \in G$, if $\phi(g) = g^\prime$, 
  then $\inv{\phi}(g^\prime) = \{ x \in G \ | \ \phi(x) = g^\prime \} = g \Ker \phi$.
\end{lemma}
\begin{proof}
  Let $x \in \inv{\phi}(g^\prime)$, 
  then $\phi(x) = g^\prime$, 
  by $\phi(g) = g^\prime$
  we have $\phi(x) = \phi(g)$
  and $x \Ker \phi = g \Ker \phi$,
  thus $x \in g \Ker \phi$.

  Let $gx \in \Ker \phi$ where $\phi(x) = e$.
  $\phi(gx) = \phi(g)\phi(x) = g^\prime e = g^\prime$,
  thus $gx \in \inv{\phi}(g^\prime)$.
\end{proof}

\begin{theorem}[Properties of Homomorphism]
  The propositions above are true.
\end{theorem}
\begin{proof}
  Trivial.
\end{proof}

The following lemmas assume $H$ is a subgroup of $G$.

\begin{theorem}[Properties of Subgroups Under Homomorphisms]
  The following propositions are true.
\end{theorem}

\begin{lemma}[Homomorphism Preserves Subgroup]
  $\phi(H)$ is a subgroup of $\phi(\overline{G})$.
\end{lemma}
\begin{proof}
  By three-steps:

  \begin{itemize}
    \item $\phi(e) \in \phi(H)$.
    \item For any $\phi(a) \ \phi(b) \in \phi(H)$, $\phi(a)\phi(b) = \phi(ab) \in \phi(H)$.
    \item For any $\phi(a) \in \phi(H)$, $\inv{\phi(a)} = \phi(\inv{a}) \in \phi(H)$.
  \end{itemize}
\end{proof}

\begin{lemma}[Homomorphism Preserves Cyclic]
  If $H$ is cyclic, then $\phi(H)$ is cyclic.
\end{lemma}
\begin{proof}
  Let $H = \langle h \rangle$. 
  For any $\phi(h^n) \in \phi(H)$,
  $\phi(h^n) = \phi(h)^n$, thus $\phi(H) = \langle \phi(h) \rangle$.
\end{proof}

\begin{lemma}[Homomorphism Preserves Abelian]
  If $H$ is Abelian, then $\phi(H)$ is Abelian.
\end{lemma}
\begin{proof}
  For any $\phi(a) \ \phi(b) \in \phi(H)$.
  $\phi(a)\phi(b) = \phi(ab) = \phi(ba) = \phi(b)\phi(a)$.
\end{proof}

\begin{lemma}[Homomorphism Preserves Normality]
  If $H \normalin G$, then $\phi(H) \normalin \phi(G)$.
\end{lemma}
\begin{proof}
  For any $\phi(g) \in \phi(G)$ and $\phi(h) \in \phi(H)$.
  \begin{align*}
     & \phi(g)\phi(h)\inv{\phi(g)} \\
    =& \phi(gh\inv{g}) \\
    =& \phi(h^\prime) \in \phi(H) && \text{(By $H \normalin G$)} \\
  \end{align*}
\end{proof}

\begin{lemma}
  If $|\Ker \phi| = n$, then $\phi$ is an $n$-to-$1$ mapping from $G$ onto $\phi(G)$.
\end{lemma}
\begin{proof}
  It is trivial that $\phi$ is $G$ onto $\phi(G)$.

  Let $g \in G$, and for any $x \in \Ker \phi$, $\phi(xg) = \phi(x)\phi(g) = \phi(g)$.
  In other words, $\phi(g \Ker \phi) = \{ \phi(g) \}$.
\end{proof}

\begin{lemma}
  If $H$ is finite, then $|\phi(H)|$ divides $|H|$.
\end{lemma}
\begin{proof}
  Let $K = H \cap \Ker \phi$, 
  since both $H$ and $\Ker \phi$ are subgroups of $G$,
  so is $K$.

  We will show that $K$ is normal in $H$.
  For any $h \in H$ and $k \in K$, 
  since $h$, $\inv{h}$ and $k$ are in $H$, so is $hk\inv{h}$.
  $\phi(hk\inv{h}) = \phi(h)\phi(k)\phi(\inv{h}) = \phi(h)e\phi(\inv{h}) = e$ so $hk\inv{h} \in \Ker \phi$.
  So $hk\inv{h} \in K$.

  Then we show $H / K \approx \phi(H)$.
  We claim the following function is an isomorphism:

  \begin{center}
    \boxed{\psi(hK) = \phi(h)}
  \end{center}

  But first of all, we need to show it \textbf{is} a function.
  For any $hK$ and $kK$ in $H/K$, and $hK = kK$.
  We have $\inv{h}k \in K$ which implies $\inv{h}k \in \Ker \phi$
  and then $h \Ker \phi = k \Ker \phi \rightarrow \phi(h) = \phi(k)$.

  \begin{itemize}
    \item One-to-one: If $\psi(hK) = \psi(kK)$, 
      then $\phi(h) = \phi(k)$ and $h \Ker \phi = k \Ker \phi$.
      Since $h$ and $k$ are in $H$, $hH = kH = H$. So:
      
      \begin{align*}
        hK &= h(H \cap \Ker \phi) \\
           &= hH \cap h \Ker \phi && \text{(Since $\lambda x. hx$ is injective)} \\
           &= kH \cap k \Ker \phi \\
           &= k (H \cap \Ker \phi) \\
           &= kK
      \end{align*}
    \item Onto: For any $\phi(h) \in \phi(H)$ for some $h$, $\psi(hK) = \phi(h)$.
    \item Structure-Preserve:
      \begin{align*}
        \psi(hKkK) &= \psi(hkK) \\
          &= \phi(hk) \\
          &= \phi(h)\phi(k) \\
          &= \psi(hK)\psi(kK)
      \end{align*}
  \end{itemize}
  
  Thus, $\displaystyle |\phi(H)| = \frac{|H|}{|K|}$.
\end{proof}

\begin{lemma}[Kernel is Normal]
  $\Ker \phi$ is normal in $G$
\end{lemma}
\begin{proof}
  We need to show: $\forall g \in G, g (\Ker\phi) \inv{g} \subseteq \Ker\phi$.
  For any $g \in G$ and $k \in \Ker\phi$, 
  $\phi(gk\inv{g}) = \phi(g)\phi(k)\phi(\inv{g}) = \phi(g)e\phi(\inv{g}) = e$.
  So $gk\inv{g} \in \Ker\phi$.
\end{proof}

\begin{lemma}
  $\phi(Z(G))$ is a subgroup of $Z(\phi(G))$.
\end{lemma}
\begin{proof}
  For any $\phi(g) \in \phi(Z(G))$ for some $g \in Z(G)$. 
  For any $\phi(a) \in \phi(G)$,
  $\phi(a)\phi(g) = \phi(ag) = \phi(ga) = \phi(g)\phi(a)$, so $\phi(g) \in Z(\phi(G))$.
  Thus $\phi(Z(G)) \subseteq Z(\phi(G))$.

  Then by three-steps:

  \begin{itemize}
    \item $\phi(e) \in \phi(Z(G))$.
    \item For any $\phi(a) \ \phi(b) \in \phi(Z(G))$ where $a \ b \in Z(G)$, $\phi(a)\phi(b) = \phi(ab) \in \phi(Z(G))$.
    \item For any $\phi(a) \in \phi(Z(G))$ where $a \in Z(G)$, $\inv{\phi(a)} = \phi(\inv{a}) \in \phi(Z(G))$.
  \end{itemize}
\end{proof}

\begin{lemma}[Inverse Image of Subgroup is Subgroup]
  If $\overline{K}$ is a subgroup of $\overline{G}$, 
  then $\inv{\phi}(\overline{K}) = \{ k \in G \ | \ \phi(k) \in \overline{K} \}$
  is a subgroup of $G$.
\end{lemma}
\begin{proof}
  It is clearly a subset of $G$.

  We will prove it is a subgroup by three-steps:
  \begin{itemize}
    \item $\phi(e) \in \overline{K}$
    \item For any $a \ b \in \inv{\phi}(\overline{K})$,
      $\phi(ab) = \phi(a)\phi(b)$ where $\phi(a) \ \phi(b) \in \overline{K}$,
      so is $\phi(ab)$.
    \item For any $a \in \inv{\phi}(\overline{K})$,
      $\phi(\inv{a}) = \inv{\phi(a)}$ where $\phi(a) \in \overline{K}$,
      so is $\inv{\phi(a)}$.
  \end{itemize}
\end{proof}

{
\newcommand{\K}{\inv{\phi}(\overline{K})}

\begin{lemma}[Inverse Image of Normal is Normal]
  If $\overline{K}$ is a normal subgroup of $G$, 
  then $\K = \{ k \in G \ | \ \phi(k) \in \overline{K} \}$ is a normal subgroup of $G$.
\end{lemma}
\begin{proof}
  We proved that $\K$ is a subgroup of $G$.

  For any $g \in G$ and $k \in \K$,
  $\phi(gk\inv{g}) = \phi(g) \phi(k) \inv{\phi(g)} = \overline{k}^\prime \in \overline{K}$ 
  since $\overline{K}$ is normal, thus $gk\inv{g} \in \K$.
\end{proof}
}

\begin{lemma}
  If $\phi : G \rightarrow \overline{G}$ is onto 
  and $\Ker \phi = \1$, then $\phi$ is an isomorphism
  from $G$ to $\overline{G}$
\end{lemma}
\begin{proof}
  We need to show $\phi$ is one-to-one.

  For any $\phi(a)$ and $\phi(b)$ such that $\phi(a) = \phi(b)$.
  We know $a \Ker \phi = b \Ker \phi$, but $\Ker \phi = \1$.
  So $\{ a \} = \{ b \}$ implies $a = b$.
\end{proof}

\begin{theorem}[First Isomorphism Theorem]
  \label{thm:10.3}
  Let $\phi : G \rightarrow \overline{G}$, 
  then the mapping from $G / \Ker \phi$ to $\phi(G)$, 
  given by $g \Ker \phi \mapsto \phi(g)$
  is an isomorphism.
\end{theorem}
\begin{proof}
  We denote that isomorphism as $\psi$.
  First, we need to show $\psi$ \textbf{is} a function.
  That is, for any $g \Ker \phi$ and $h \Ker \phi$,
  if $g \Ker \phi = h \Ker \phi$,
  then $\psi(g \Ker \phi) = \psi(h \Ker \phi)$.

  If $g \Ker \phi = h \ker \phi$, by Lemma \ref{lemma:10.7},
  we know $\phi(g) = \phi(h)$ and then $\psi(g \Ker \phi) = \psi(h \Ker \phi)$.

  Then we need to show it is an isomorphism:
  \begin{itemize}
    \item One-to-one: If $\psi(g \Ker \phi) = \psi(h \Ker \psi)$,
      $\phi(g) = \phi(h)$ gives $g \Ker \phi = h \Ker \phi$.
    \item  Onto: For any $\phi(g) \in \phi(G)$ for some $g \in G$,
      we have $\psi(g \Ker \phi) = \phi(g)$.
    \item Structure-Preserve: For any $g \Ker \phi$ and $h \Ker \phi$:
      \begin{align*}
         & \psi(g\Ker\phi h\Ker\phi) \\
        =& \psi(gh\Ker\phi) \\
        =& \phi(gh) \\
        =& \phi(g) \phi(h) \\
        =& \psi(g\Ker\phi) \psi(h\Ker\phi)
      \end{align*}
  \end{itemize}
\end{proof}

\begin{lemma}[N/C Theorem]
  Suppose $H$ is a subgroup of $G$,
  the normalizer $N(H) = \{ x \in G \ | \ xH\inv{x} = H \}$,
  and the centerlizer $C(H) = \{ x \in G \ | \ \forall h \in H, xh = hx \}$
  (or equivalently, $\{ C(H) = \{ x \in G \ | \ \forall h \in H, xh\inv{x} = h\} \}$).
  Consider the homomorphism $\psi(g) = \phi_g : N(H) \rightarrow \Aut(H)$.
  $N(H)/C(H)$ is isomorphic to some subgroup of $\Aut(H)$.
\end{lemma}
\begin{proof}
  It is easy to show that $C(H) = \Ker \psi$.
  Since for any $g \in N(H)$, $\psi(g) = \phi_e \rightarrow \phi_g = \phi_e$,
  then, $g$ commutes with any $h \in H$, thus $g \in C(H)$. 
  And for any $g \in C(H)$, $\psi(g) = \phi_g$, 
  since $g$ commutes with any $x \in H$, we have $\phi_g(x) = gx\inv{g} = xg\inv{g} = x$.
  This tell us that $\psi(g) = \phi_g = \phi_e$, thus $\psi(h) \in \Ker \psi$.

  Thus $N(H)/C(H) = N(H)/\Ker \psi$, by Theorem \ref{thm:10.3}, 
  $N(H)/C(H) \approx \psi(H)$ which is a subgroup of $\Aut(H)$.
\end{proof}

\begin{theorem}[Normal Subgroups are Kernels]
  Let $H \normalin G$, 
  $H$ is a kernel of some homomorphism of $G$.
  In particular, $H$ is a kernel of the mapping $\gamma(g) = gH$ from $G$ to $G/H$.
\end{theorem}
\begin{proof}
  First, we need to show $\gamma$ is a homomorphism.
  For any $a \ b \in G, \gamma(ab) = abH = aHbH = \gamma(a)\gamma(b)$.

  For any $h \in H$, $\gamma(h) = hH = H$ because $h \in H$,
  since $H$ is the identity of $G/H$, $H \subseteq \Ker \gamma$.

  For any $x \in \Ker \gamma$, we know $\gamma(x) = xH = H$,
  this implies $x \in H$, thus $\Ker \gamma \subseteq H$.
\end{proof}
  
\begin{lemma}[Composition of Homomorphism]
  Let $\phi : G \rightarrow H$ and $\psi : H \rightarrow K$ are homomorphisms.
  $\psi \circ \phi : G \rightarrow K$  is also homomorphism.
\end{lemma}
\begin{proof}
  For any $a \ b \in G$:
  \begin{align*}
    (\psi \circ \phi)(ab) &= \psi(\phi(ab)) \\
    &= \psi(\phi(a) \phi(b)) && \text{(By $\phi$ is homomorphism)} \\
    &= \psi(\phi(a)) \psi(\phi(b)) && \text{(By $\psi$ is homomorphism)} \\
    &= (\psi \circ \phi)(a) (\psi \circ \phi)(b)
  \end{align*}
\end{proof}

\begin{lemma}[Restricted Homomorphism]
  Let $\phi : G \rightarrow \overline{G}$ and $H$ a subgroup of $G$.
  If $\Ker \phi \subseteq H$, prove that $\psi : H \rightarrow \overline{G}$
  given by $\psi(h) = \phi(h)$ is also a homomorphism with kernel $\Ker \phi$.
\end{lemma}
\begin{proof}
  $\psi$ preserve structure by directly calls $\phi$.
  For any $k \in \Ker \phi$, since $\Ker \phi \subseteq H$,
  $\psi(k) = \phi(k) = e$.
  And for any $h \in H$ such that $\psi(h) = = e$, $h \in \Ker \phi$
  since $\psi(h) = \phi(h) = e$.
\end{proof}

\begin{lemma}[Homomorphism on Internal Direct Product is False]
  The following proposition is False:
  Let $\phi$ a homomorphism from $G$ to some group and $G = H \times K$.
  Show that $\phi(H \times K) = \phi(H) \times \phi(K)$.
\end{lemma}
\begin{proof}
  Consider $G = Z_2 \oplus Z_2$ and homomorphism $\phi((a , b)) = a + b$.
  Obviously $\langle (1 , 1) \rangle$ is the kernel, and
  $Z_2 \oplus Z_2 = (Z_2 \oplus \1) \times (\1 \oplus Z_2)$.
  If the hypothesis is true, then 
  the intersection of $\phi(Z_2 \oplus \1) = Z_2$ and $\phi(\1 \oplus Z_2) = Z_2$
  should be $\1$ but it is not.
\end{proof}

\end{document}