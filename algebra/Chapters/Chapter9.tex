\documentclass[14pt]{extarticle}
\usepackage[T1]{fontenc}
\usepackage[margin=1in]{geometry}
\usepackage{amsthm,amsmath}

\newtheorem{theorem}{Theorem}[section]
\newtheorem{lemma}{Lemma}[section]
\newtheorem*{example}{Example}
\newtheorem{definition}{Definition}[section]
\setcounter{section}{9}

\newcommand{\inv}[1]{#1^{-1}}
\newcommand{\join}[3][,]{#2_0 #1 #2_1 #1 \cdots #1 #2_{#3}}

\begin{document}

\begin{definition}[Normal Subgroup]
  A subgroup $H$ of $G$ is normal if for any $a \in G$, $aH = Ha$.
\end{definition}

\begin{theorem}[Normal Subgroup Test]
  \label{9.1}
  A subgroup $H$ of $G$ is normal in G $\iff$ $xH\inv{x} \subseteq H$ for all $x$ in $G$.
\end{theorem}

\begin{proof}
  We must show fisrt:
  \begin{center}
    \fbox{
      The subgroup $H \subseteq G$ is normal $\implies$
      $\forall x \in G, xH\inv{x} \subseteq H$
    }
  \end{center}

  By the definition of normal, we know $\forall a \in G, aH = Ha$,
  that is, $\forall a \in G, h \in H, \exists h', ah = h'a$.
  Therefore, for all elements $xh\inv{x}$ in $xH\inv{x}$,
  $xh\inv{x} = h'x\inv{x} = h'e = h$ which is in $H$.

  So $xH\inv{x} \subseteq H$.

  Secondly, we must show:

  \begin{center}
    \fbox{
      $\forall x \in G, xH\inv{x} \subseteq H \implies$ the subgroup $H \subseteq G$ is normal
    }
  \end{center}

  By the definition of normal, we need to show $\forall a \in G, aH = Ha$,
  or equivalently, $aH \subseteq Ha$ and $Ha \subseteq aH$.
  
  By the hypothesis, we know $aH\inv{a} \subseteq H$, that is,
  $\forall h \in H, ah\inv{a} \in H$.

  By multiplying $a$ at right side, we get $(ah\inv{a})a = ah(\inv{a}a) = ah \in Ha$
  where $ah \in aH$. And we finished the left half part of our goal. 
  By taking $\inv{a}$ as the argument of hypothesis, 
  we can finish the right half part in a similar way.
\end{proof}

\begin{example}
  Let $H$ be a normal subgroup of $G$ and $K$ be any subgroup of $G$. 
  Shows that $HK = \{ hk \ | \ h \in H, k \in K \}$ is a subgroup of $G$.
\end{example}
\begin{proof}
  By two-steps test of subgroup:
  \begin{enumerate}
    \item $ee \in HK$, thus $HK$ is not empty.
    \item $\forall h_0k_0 \ h_1k_1 \in HK$,
      $h_0k_0h_1k_1 = h_0(k_0h_1)k_1 = h_0(h_1'k_0)k_1 = h_0h_1'k_0k_1 \in HK$,
      thus $HK$ is closed under multiplication.
    \item $\forall hk \in HK$, $\inv{(hk)} = \inv{k}\inv{h} = (\inv{h})^\prime \inv{k} \in HK$.
  \end{enumerate}
\end{proof}

\begin{theorem}[Quotient Group]
  Let $G$ a group and $H$ a normal group of $G$, 
  the quotient group $G/H = \{ aH \ | \ a \in G \}$ is a group under multiplication
  $\forall aH \ bH \in G/H \mapsto (aH)(bH)$.
\end{theorem}

\begin{proof}
  Before our proof, we need to show an important property of this multiplication:

  \begin{center}
    \boxed{\forall aH \ bH \in G, (aH)(bH) = abH}
  \end{center}

  For all $h_0 \ h_1 \in H$, $ah_0bh_1 = a(h_0b)h_1 = a(bh_0^\prime)h_1 = abh_0^\prime h_1 \in abH$.
  Also, for all $h \in H$, $abh = aebh \in (aH)(bH)$.

  Group axioms:
  \begin{itemize}
    \item $eH$ is the identity:
      \begin{itemize}
        \item $\forall aH \in G/H, aHeH = aeH = aH$
        \item $\forall aH \in G/H, eHaH = eaH = aH$
      \end{itemize}
    \item For all $aH \in G/H$, $\inv{a}H$ is the inverse of $aH$:
      \begin{itemize}
        \item $aH\inv{a}H = a\inv{a}H = eH$
        \item $\inv{a}HaH = \inv{a}aH = eH$
      \end{itemize}
    \item For all $aH \ bH \ cH \in G/H$, the associativity: \\
      $(aHbH)cH = abHcH = (ab)cH = a(bc)H = aHbcH = aH(bHcH)$
  \end{itemize}
\end{proof}

\begin{theorem}[G/Z]
  Let $G$ a group, $Z(G)$ the center of $G$, if $G/Z(G)$ is cyclic, $G$ is Abelian.
\end{theorem}
\begin{proof}
  Let $b \ c \in G$ but $b \ c \notin Z(G)$. If no such $b$ or $c$, then $G$ is Abelian.
  Since $G/Z(G)$ is cyclic, $G/Z(G) = \langle aZ(G) \rangle$ for some $a$.
  $b$ and $c$ must in some coset, say, $b \in a^mZ(G)$ and $c \in a^nZ(G)$.
  \begin{align*}
    bc &= a^m g a^n g^\prime \\
       &= a^m a^n g g^\prime && \text{($g \in Z(G)$)} \\
       &= a^n a^m g g^\prime && \text{(By property of power)} \\
       &= a^n a^m g^\prime g && \text{($g \in Z(G)$)} \\
       &= a^n g^\prime a^m g && \text{($g \in Z(G)$)} \\
       &= cb                 && \text{(By definition)}
  \end{align*}

  Thus, $b$ and $c$ commute.
\end{proof}

\begin{lemma}
  \label{lemma9.1}
  Suppose $G$ is a finite group, and $H$ is a normal subgroup of $G$.
  If $| aH \in G/H | = n$. Then there is an element of order $n$ in $G$.
\end{lemma}
\begin{proof}
  Since $| aH | = n$, $(aH)^n = a^nH = H$. Suppose $| a | = k$,
  we have $(aH)^k = a^kH = H$, thus $|aH| = n$ divides $k$.
  So we have $| a^{\frac{k}{n}} | = n$.
\end{proof}

\begin{theorem}[$G/Z(G) \approx Inn(G)$]
  For any group $G$, $G/Z(G)$ is isomorphic to 
  $Inn(G) = \{ \phi_g(x) = gx\inv{g} \mid g \in G \}$.
\end{theorem}
\begin{proof}
  We claim the following function is an isomorphism:
  \begin{center}
    \boxed{\psi(aZ(G)) = \phi_a}
  \end{center} 

  However, we must show that it \textbf{is} a function. 
  That is, for all $a \ b \in G$, $aZ(G) = bZ(G) \implies \psi(aZ(G)) = \psi(bZ(G))$.
  Since $aZ(G) = bZ(G)$, by property of coset, $a \in bZ(G)$. 
  Then $\forall x, \phi_a(x) = ax\inv{a} = (bg)x\inv{(bg)} = bgx\inv{g}\inv{b} = bx\inv{b} = \phi_b(x)$.
  Thus $\psi$ is a function.

  \begin{itemize}
    \item One-to-one: $\forall aZ(G) \ bZ(G) \in G/Z(G), \psi(aZ(G)) = \psi(bZ(G))$.
      \begin{align*}
        \psi(aZ(G)) &= \psi(bZ(G)) \\
        \phi_a &= \phi_b \\
        ax\inv{a} &= bx\inv{b} \quad \text{(introduce $x$)} \\ 
        \inv{b}ax &= x \inv{b}a \\
      \end{align*}
      Thus $\inv{b}a \in Z(G)$, by property of coset, $aZ(G) = bZ(G)$.
    \item Onto: $\forall \phi_g \in Inn(G)$, $\psi(gZ(G)) = \phi_g$
    \item Structure-Preserve: $\forall aZ(G) \ bZ(G) \in G/Z(G)$
      \begin{align*}
        \psi(aZ(G)bZ(G)) 
          &= \psi(abZ(G)) \\
          &= \phi_{ab} \\
          &= \phi_{a} \circ \phi{b} \\
          &= \psi(aZ(G)) \circ \psi(bZ(G))
      \end{align*}
  \end{itemize}
\end{proof}

\begin{theorem}[Cauchy's Theorem]
  Let $G$ be finite Abelian group and let $p$ be a prime where $p$ divides $|G|$.
  Then there is an element of order $p$ in $G$.
\end{theorem}
\begin{proof}
  Induction on the number of prime factors of $|G|$:
  \begin{itemize}
    \item Base: We must show that $\forall G, G \ \text{is Abelian}, |G| = p$ where $p$ is prime has an element of order $p$.
      Since $|G| = p$, $G$ must be cyclic, then the order of the generator of $G$ is $p$.
    \item Induction: We have the induction hypothesis: 
      $\forall G$ a Abelian group, $|G| =$ product of $n$ primes,
      $p$ divides $|G|$,
      $\exists x \in G, |x| = p$.

      Suppose $g \in G$, if $p$ divides $|g|$, then $|g^\frac{|g|}{p}| = p$, 
      we assume that $p$ doesn't divide $|g|$.
      Let $q$ a prime that divides $|g|$, then $|g^\frac{|g|}{q}| = q$.
      We let $h = g^\frac{|g|}{q}$, and $H = \langle h \rangle$.
      Since $G$ is Abelian, $\langle h \rangle$ is a normal subgroup.
      Consider $G/H$, since $G$ is abelian, so is $G/H$,
      and $|G/H| = \frac{|G|}{|H|}$ which is the product of $n + 1 - 1 = n$ primes.
      Also, $p$ divides $G$ but not divides $H$, so $p$ divides $G/H$.
      Then by induction hypothesis, we know $\exists x \in G/H, |x| = p$.
      By Lemma \ref{lemma9.1}, $\exists x \in G, |x| = p$.
  \end{itemize}
\end{proof}

\begin{definition}[Internal Direct Product]
  Let $\{ H_0 , H_1 , H_2 \cdots H_{n-1} \}$ be a finite collection of normal subgroups of $G$.
  We say $G$ is an internal direct product of $\{ H_0 , H_1 , H_2 \cdots H_{n-1} \}$
  (write $G = H_0 \times H_1 \times \cdots \times H_{n-1}$) if:
  \begin{itemize}
    \item $G = H_0H_1 \cdots H_{n-1} = \{ h_0h_1 \cdots h_{n-1} \ | \ h_i \in H_i \}$
    \item $(H_0H_1 \cdots H_{i - 1}) \cap H_i = \{ e \}$ for all $0 \le i < n$
  \end{itemize}
\end{definition}

\begin{lemma}[Unique Representation of Internal Direct Product]
  \label{lemma9.2}
  For all $h = h_0h_1 \cdots h_{n - 1}$ and $h^\prime = h_0^\prime h_1^\prime \cdots h_{n - 1}^\prime$,
  $h = h^\prime$ implies $h_i = h_i^\prime$ for all $0 \le i < n$
\end{lemma}
\begin{proof}
  Induction on $n$:
  \begin{itemize}
    \item Base: We must show $h_0 = h_0^\prime$ implies $h_0 = h_0^\prime$ which is trivial.
    \item Induction: We have the following induction hypothesis:
      \begin{center}
        \boxed{h_0h_1 \cdots h_{i - 1} = h_0^\prime h_1^\prime \cdots h_{i - 1}^\prime \implies h_j = h_j^\prime \quad (\forall 0 \le j < i)}
      \end{center}
      and we must show:
      \begin{center}
        \boxed{h_0h_1 \cdots h_i = h_0^\prime h_1^\prime \cdots h_i^\prime \implies h_j = h_j^\prime \quad (\forall 0 \le j \le i)}
      \end{center}

      Let $h = h_0h_1 \cdots h_{i - 1}$ and $h^\prime = h_0^\prime h_1^\prime \cdots h_{i - 1}^\prime$,
      $hh_i = h^\prime h_i^\prime$ gives $h = h^\prime h_i^\prime \inv{h_i}$,
      where $h \ h^\prime \in H_0H_1 \cdots H_{i - 1}$ and $\inv{h_i} \ h_i^\prime \in H_i$.
      Then $h^\prime h_i^\prime \inv{h_i} \in H_0H_1 \cdots H_{i - 1}$.
      By the property of group multiplication, $h_i^\prime \inv{h_i} \in H_0H_1 \cdots H_{i - 1}$.
      But by the property of internal direct product, $(H_0H_1 \cdots H_{i - 1}) \cap H_i = \{ e \}$.
      So $h_i^\prime \inv{h_i} = e \rightarrow h_i = h_i^\prime$, 
      $hh_i = h^\prime h_i^\prime \rightarrow h = h^\prime$.
      By induction hypothesis, $h_j = h_j^\prime \quad (\forall 0 \le j < i)$.
  \end{itemize}
\end{proof}

\begin{example}
  Let $G$ is the internal direct product of subgroups $H_0, H_1 \cdots H_{n - 1}$.
  Let $h_i \in H_i$, $h_j \in H_j$ where $0 \le i, j < n$. $h_ih_j = h_jh_i$ if $i \neq j$.
\end{example}
\begin{proof}
  By property of normal, $h_i^\prime h_j = h_jh_i$ = $h_ih_j^\prime$,
  then by Lemma \ref{lemma9.2}, $h_i = h_i^\prime$ $h_j = h_j^\prime$.
  Thus $h_jh_i = h_ih_j^\prime = h_i h_j$
\end{proof}

\begin{lemma}[Center of $h \in H$]
  Let $G$ be the internal direct product of subgroups $\join{H}{n - 1}$,
  $\forall \ h \in H_i, \prod_{j = 0, i \neq j} H_j$ is a subgroup of $C(h)$.
\end{lemma}
\begin{proof}
  Since for any normal subgroup $H$ of $G$, and any subgroup $K$ of $G$, $HK$ is a subgroup of $G$,
  then $\prod_{j = 0, i \neq j} H_j$ is a subgroup of $G$.
  Consider $k \in H_s$ for some $s$.
  By property of normal subgroup $H_s$, $hk = k^\prime h$ for some $k^\prime$.
  Also, for normal subgroup $H_i$, $hk = kh^\prime$ for some $h^\prime$.
  Then $hk = k^\prime h = k h^\prime$, by Lemma \ref{lemma9.2},
  $h = h^\prime$ and $k = k^\prime$.
  So $hk = k^\prime h = kh$, $k \in C(h)$.
  Now we consider any element $k$ in $\prod_{j = 0, i \neq j} H_j$,
  it must be the product of elements $k_s$ in the corresponding $H_s$,
  $h$ commutes with all the $k_s$, so does $k$.
  Thus, $\forall \ k \in \prod_{j = 0, i \neq j} H_j, k \in C(h)$.
\end{proof}

\begin{theorem}[Internal Direct Product $\approx$ External Direct Product]
  If $G$ is the internal direct product of subgroups $H_0, H_1 \cdots H_{n - 1}$,
  then $H_0 \times H_1 \times \cdots \times H_{n - 1} \approx H_0 \oplus H_1 \oplus \cdots \oplus H_{n - 1}$.
\end{theorem}
\begin{proof}
  We claim the following function:
  \begin{center}
    \boxed{\phi(h_0h_1 \cdots h_{n - 1}) = (h_0, h_1, \cdots, h_{n - 1})}
  \end{center}
  is an isomorphism, and 
  by Lemma \ref{lemma9.2}, we know $\phi$ \textbf{is} a function.

  \begin{itemize}
    \item One-to-one: trivial.
    \item Onto: trivial.
    \item Structure-Preserve:
      \begin{align*}
         & \phi(h_0 h_1 \cdots h_{n - 1}) \phi(h_0^\prime h_1^\prime \cdots h_{n - 1}^\prime) \\
        =& (h_0, h_1, \cdots h_{n - 1}) (h_0^\prime, h_1^\prime, \cdots h_{n - 1}^\prime) \\
        =& (h_0h_0^\prime, h_1h_1^\prime, \cdots, h_{n - 1}h_{n - 1}^\prime) \\
        =& \phi(h_0h_0^\prime h_1h_1^\prime \cdots h_{n - 1}h_{n - 1}^\prime) \\
        =& \phi(h_0h_1h_1^\prime\cdots h_{n - 1} h_0^\prime h_{n - 1}^\prime) \\
        =& \cdots \\
        =& \phi(\join[]{h}{n - 1} \join[]{h^\prime}{n - 1}) \\
      \end{align*}
  \end{itemize}
\end{proof}

\begin{lemma}[Normal Subgroup of External Direct Product]
  For all $G = H_0 \oplus H_1$, $H_0 \oplus \{ e \}$ and $\{ e \} \oplus H_1$
  are normal subgroups of $G$.
\end{lemma}
\begin{proof}
  $\forall \ (a , b) \in H_0 \oplus H_1, (h , e) \in H_0 \oplus \{ e \}$,
  $(a , b) (h , e) (\inv{a} , \inv{b}) = (a h \inv{a} , b \inv{b}) = (a h \inv{a} , e) \in H_0 \oplus \{ e \}$.
  Similarly for $\{ e \} \oplus H_1$.
\end{proof}

\begin{lemma}[Isomorphism respect normal]
  For any group $G \ \overline{G}$ and normal subgroup $H$ of $G$,
  if $\phi : G \approx \overline{G}$, then $\phi(H)$ is normal.
\end{lemma}
\begin{proof}
  We need to show $\forall \ g \in \overline{G}, g\phi(H)\inv{g} \in \phi(H)$.
  Since $\inv{\phi}(g)H\inv{\phi}(\inv{g}) \in H$, by applying $\phi$,
  we get $g\phi(H)\inv{g} \in \phi(H)$.
\end{proof}

\begin{lemma}[Isomorphism is congruent on Quotient]
  For all group $G \ \overline{G}$ and normal subgroup $H$ of $G$. If $\phi : G \approx \overline{G}$,
  then $G/H \approx \overline{G}/\phi(H)$.
\end{lemma}
\begin{proof}
  We claim the following function is an isomorphism:

  \begin{center}
    \boxed{\psi(gH) = \phi(g)\phi(H)}
  \end{center}

  We need to show the function we claim \textbf{is} a function.
  $\forall \ gH \ g^\prime H \in G/H$, $gH = g^\prime H$, shows that $\psi(gH) = \psi(g^\prime H)$.
  Since $gH = g^\prime H$, $\inv{g}g^\prime \in H$, 
  then $\phi(\inv{g}g^\prime) \in \phi(H) \rightarrow \phi(\inv{g})\phi(g^\prime) \in \phi(H) \rightarrow \phi(g)\phi(H) = \phi(g^\prime)\phi(H)$.

  \begin{itemize}
    \item One-to-one: $\psi(gH) = \psi(g^\prime H) \rightarrow \phi(g)\phi(H) = \phi(g^\prime)\phi(H)$, then
      \begin{align*}
         & \ \phi(\inv{g})\phi(g^\prime) \in \phi(H) \\
        \rightarrow& \ \inv{\phi}(\phi(\inv{g})\phi(g^\prime)) \\
        =& \ \inv{\phi}(\phi(\inv{g}))\inv{\phi}(\phi(g^\prime)) \\ 
        =& \ \inv{g}g^\prime \in H
      \end{align*}
      which implies $gH = g^\prime H$.
    \item Onto: $\forall \ g\phi(H) \in \overline{G}/\phi(H)$
      \begin{align*}
         & \psi(\inv{\phi}(g\phi(H))) \\
        =& \psi(\inv{\phi}(g)\inv{\phi}(\phi(H))) \\
        =& \psi(\inv{\phi}(g)H) \\
        =& \phi(\inv{\phi}(g))\phi(H) \\
        =& g\phi(H)
      \end{align*}
    \item Structure-Preserve:
      \begin{align*}
         & \psi(gH g^\prime H) \\
        =& \psi(gg^\prime H) \\
        =& \phi(gg^\prime) \phi(H) \\
        =& \phi(g)\phi(g^\prime) \phi(H) \\
        =& \phi(g)\phi(H) \phi(g^\prime)\phi(H) \\
        =& \psi(gH) \psi(g^\prime H)
      \end{align*}
  \end{itemize}
\end{proof}

\begin{lemma}[Quotien Group of External Direct Product]
  $(H_0 \oplus H_1) / (H_0 \oplus \{ e \}) \approx H_1$ and $(H_0 \oplus H_1) / (H_1 \oplus \{ e \}) \approx H_0$.
\end{lemma}
\begin{proof}
  {
    \newcommand{\coset}{(H_0 \oplus \{ e \})}

    We claim the following function is an isomorphism:

    \begin{center}
      \boxed{\phi((e , h)\coset) = h}
    \end{center}

    But first, we need to show it \textbf{is} a function.
    We need to show that for any $h\coset \in (H_0 \oplus H_1)/\coset$ where $h \in H_0 \oplus H_1$,
    $\phi$ is defined at $h\coset$. Since $h = (h_0 , h_1) = (e , h_1) (h_0 , e)$,
    $h\coset = (e , h_1) (h_0 , e) \coset$ where $(h_0 , e) \in \coset$.
    Thus $h\coset = (e , h_1) \coset$. Then we need to show that
    $(e , h_0)\coset = (e , h_1)\coset \implies \phi((e , h_0)\coset) = \phi((e, h_1)\coset)$.
    It is clearly that any element in $(e , h_0)\coset$ has form $(a , h_0)$.
    Similarly, in $(e , h_1)\coset$ it is $(a , h_1)$. Thus $h_0 = h_1$.

    \begin{itemize}
      \item One-to-one: Trivial.
      \item Onto: Trivial.
      \item Structure-Preserve:
        \begin{align*}
           & \phi((e , h_0)\coset (e , h_1)\coset) \\
          =& \phi((e , h_0 h_1)\coset) \\
          =& h_0 h_1 \\
          =& \phi((e , h_0)\coset) \phi((e , h_1)\coset)
        \end{align*}
    \end{itemize}
  }

  For $\{ e \} \oplus H_1$, we observe that 
  \begin{align*}
     & (H_0 \oplus H_1) / (\{ e \} \oplus H_1) \\
    \approx & (H_1 \oplus H_0) / (H_1 \oplus \{ e \}) \\
    \approx & H_0
  \end{align*}
\end{proof}

\begin{lemma}[Cancellation of Internal Direct Product is False]
  The following proposition is False: 
  Let $G = H \times K$ and $G = H^\prime \times K$, then $H = H^\prime$.
\end{lemma}
\begin{proof}
  Consider $Z_2 \oplus Z_2$, 
  we have $(Z_2 \oplus \{ e \}) \times (\{ e \} \oplus Z_2) = Z_2 \oplus Z_2$,
  and $\langle (1 , 1) \rangle \times (\{ e \} \oplus Z_2) = Z_2 \oplus Z_2$.
  But $Z_2 \oplus \{ e \} \neq \langle (1 , 1) \rangle$.
\end{proof}

\end{document}