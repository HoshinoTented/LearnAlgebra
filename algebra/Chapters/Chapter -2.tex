\documentclass[14pt]{extarticle}
\usepackage[T1]{fontenc}
\usepackage[margin=1in]{geometry}
\usepackage{amsthm,amsmath}
\usepackage{hyperref}

\newtheorem{theorem}{Theorem}[section]
\newtheorem{lemma}{Lemma}[section]
\newtheorem{definition}{Definition}[section]
\newtheorem{exercise}{Exercise}[section]
\newtheorem*{example}{Example}
\setcounter{section}{-2}

\newcommand{\inv}[1]{#1^{-1}}
\newcommand{\join}[3][,]{#2_0 #1 #2_1 #1 \cdots #1 #2_{#3}}
\newcommand{\Times}[2]{\join[\times]{#1}{#2}}
\newcommand{\Oplus}[2]{\join[\oplus]{#1}{#2}}
\newcommand{\normalin}{\triangleleft}
\newcommand{\1}{\{e\}}
\newcommand{\set}[2]{\{ \ #1 \ | \ #2 \ \}}
\newcommand{\cyc}[1]{\langle #1 \rangle}

\DeclareMathOperator{\Abelian}{Abelian}
\DeclareMathOperator{\Inn}{Inn}
\DeclareMathOperator{\Aut}{Aut}
\DeclareMathOperator{\Ker}{Ker}
\DeclareMathOperator{\modu}{mod}
\DeclareMathOperator{\id}{id}
\DeclareMathOperator{\lcm}{lcm}

\begin{document}

This chapter established the set theory of hoshino version.

\begin{definition}[Minimum]
  Let $S$ a set and $n \in S$, $n$ is minimum if
  for any $m \in S$, $n \leq m$.
\end{definition}

\begin{theorem}
  Let $S$ be a non-empty set which consists of natural number,
  show that there is $n \in S$ such that $n$ is minimum.
\end{theorem}
\begin{proof}
  Suppose there no such $n \in S$ where $n$ is minimum, then
  for any $n \in S$, $n$ is not minimum,
  then for any $n \in S$, there is $m \in S$ such that $n > m$.
  Therefore, for any $n \in S$, we can obtain a smaller element $m$.

  Let $n \in S$, then we can get a smaller element $m$,
  and do the same thing on $m$. We will finally reach $0 \in S$,
  nut here is no any natural number that smaller than $0$,
  but we can still obtain a $m$ such that $m < 0$,
  this is unacceptible.

  So $S$ has a smallest element.
\end{proof}

\begin{theorem}
  Let $S$ a set and $S_i$ a collection of set, and $C_i = S \setminus S_i$ a collection of
  complements. Then
  \[
  \bigcup C_i = \bigcup (S \setminus S_i) = S \setminus (\bigcap S_i)
  \]
\end{theorem}
\begin{proof}
  $(\supseteq)$ For any element $x \in \bigcup C_i$, we know $x$ must belongs to some $C_\alpha$,
  therefore $x \notin S \setminus C_\alpha = S_\alpha$, so $x \notin \bigcap S_i$,
  therefore $x \in S \setminus (\bigcap S_i)$.

  $(\subseteq)$ For any element $x \in S \setminus (\bigcap S_i)$, we know $x \in S$
  but $s \notin \bigcap S_i$, therefore, $x$ must not belongs to some $S_\alpha$,
  therefore $x \in S \setminus S_\alpha = C_\alpha$, so $x \in \bigcup C_i$.
\end{proof}

\begin{theorem}
  Let $S$ a set and $S_i$ a collection of set, and $C_i = S \setminus S_i$ a collection of complements.
  Then
  \[
  \bigcap C_i = \bigcap (S \setminus S_i) = S \setminus (\bigcup S_i)
  \]
\end{theorem}
\begin{proof}
  $(\supseteq)$ For any element $x \in \bigcap C_i$, we know $x$ belongs to all $C_i$,
  for any $S_i = S \setminus C_i$, $x \notin S_i$, so $x \notin \bigcup S_i$,
  therefore $s \in S \setminus (\bigcup S_i)$.
  $(\subseteq)$ For any element $x \in S \setminus (\bigcup S_i)$, we know $x \in S$
  but $x \notin \bigcup S_i$, so none of them contains $x$, therefore every $S \setminus S_i$ contains $x$,
  that is, $x \in \bigcap (S \setminus S_i) = \bigcap C_i$. 
\end{proof}

\begin{theorem}
  Suppose $f : A \rightarrow B$, then for any $V, W \subseteq A$
  we have:
  \begin{itemize}
    \item $f(V \cap W) \subseteq f(V) \cap f(W)$
    \item $f(V \cup W) = f(V) \cup f(W)$
  \end{itemize}
\end{theorem}
\begin{proof}
  ~
  \begin{itemize}
    \item For any $x \in f(V \cap W)$, we there is $y \in V \cap W$ such that $f(y) = x$,
          then $x \in f(V)$ since $y \in V$ and $x \in f(W)$ since $y \in W$.
    \item $(\supseteq)$ For any $x \in f(V) \cup f(W)$, we know $x \in f(V)$ or $x \in f(W)$,
          we may suppose $x \in f(V)$. Then there is $y \in V$ such that $f(y) = x$,
          therefore $x \in f(V \cup W)$ since $y \in V \cup W$.

          $(\subseteq)$ For any $x \in f(V) \cup f(W)$, we may suppose $x \in f(V)$.
          Then there is $a \in V$ such that $f(a) = x$, then $a \in V \cup W$ and
          $x \in f(V \cup W)$.
  \end{itemize}
\end{proof}

\end{document}