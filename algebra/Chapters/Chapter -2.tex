\documentclass[14pt]{extarticle}
\usepackage[T1]{fontenc}
\usepackage[margin=1in]{geometry}
\usepackage{amsthm,amsmath}
\usepackage{hyperref}

\newtheorem{theorem}{Theorem}[section]
\newtheorem{lemma}{Lemma}[section]
\newtheorem{definition}{Definition}[section]
\newtheorem{exercise}{Exercise}[section]
\newtheorem*{example}{Example}
\setcounter{section}{-2}

\newcommand{\inv}[1]{#1^{-1}}
\newcommand{\join}[3][,]{#2_0 #1 #2_1 #1 \cdots #1 #2_{#3}}
\newcommand{\Times}[2]{\join[\times]{#1}{#2}}
\newcommand{\Oplus}[2]{\join[\oplus]{#1}{#2}}
\newcommand{\normalin}{\triangleleft}
\newcommand{\1}{\{e\}}
\newcommand{\set}[2]{\{ \ #1 \ | \ #2 \ \}}
\newcommand{\cyc}[1]{\langle #1 \rangle}

\DeclareMathOperator{\Abelian}{Abelian}
\DeclareMathOperator{\Inn}{Inn}
\DeclareMathOperator{\Aut}{Aut}
\DeclareMathOperator{\Ker}{Ker}
\DeclareMathOperator{\modu}{mod}
\DeclareMathOperator{\id}{id}
\DeclareMathOperator{\lcm}{lcm}

\begin{document}

This chapter established the set theory of hoshino version.

\begin{definition}[Minimum]
  Let $S$ a set and $n \in S$, $n$ is minimum if
  for any $m \in S$, $n \leq m$.
\end{definition}

\begin{theorem}
  Let $S$ be a non-empty set which consists of natural number,
  show that there is $n \in S$ such that $n$ is minimum.
\end{theorem}
\begin{proof}
  Suppose there no such $n \in S$ where $n$ is minimum, then
  for any $n \in S$, $n$ is not minimum,
  then for any $n \in S$, there is $m \in S$ such that $n > m$.
  Therefore, for any $n \in S$, we can obtain a smaller element $m$.

  Let $n \in S$, then we can get a smaller element $m$,
  and do the same thing on $m$. We will finally reach $0 \in S$,
  nut here is no any natural number that smaller than $0$,
  but we can still obtain a $m$ such that $m < 0$,
  this is unacceptible.

  So $S$ has a smallest element.
\end{proof}

\end{document}