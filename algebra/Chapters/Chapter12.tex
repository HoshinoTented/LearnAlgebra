\documentclass[14pt]{extarticle}

\usepackage[T1]{fontenc}
\usepackage[margin=1in]{geometry}
\usepackage{amsthm,amsmath,amssymb}

\usepackage{subfiles}

\newtheorem{theorem}{Theorem}[section]
\newtheorem{lemma}{Lemma}[section]
\newtheorem{definition}{Definition}[section]
\newtheorem{exercise}{Exercise}[section]
\newtheorem*{example}{Example}

\newcommand{\inv}[1]{#1^{-1}}
\newcommand{\join}[3][,]{#2_0 #1 #2_1 #1 \cdots #1 #2_{#3}}
\newcommand{\Times}[2]{\join[\times]{#1}{#2}}
\newcommand{\Oplus}[2]{\join[\oplus]{#1}{#2}}
\newcommand{\normalin}{\triangleleft}
\newcommand{\1}{\{e\}}
\newcommand{\set}[2]{\{ \ #1 \ | \ #2 \ \}}
\newcommand{\cyc}[1]{\langle #1 \rangle}

\DeclareMathOperator{\Abelian}{Abelian}
\DeclareMathOperator{\Inn}{Inn}
\DeclareMathOperator{\Aut}{Aut}
\DeclareMathOperator{\Ker}{Ker}
\DeclareMathOperator{\modu}{mod}
\DeclareMathOperator{\id}{id}
\DeclareMathOperator{\lcm}{lcm}

\setcounter{section}{12}

\begin{document}

\begin{definition}[Ring]
  A ring $R$ is a set with two binary operations:
  addition (denote by $a + b$) and
  multiplication (denote by $ab$), such that
  for any $a \ b \ c \in R$:
  \begin{enumerate}
    \item $a + b = b + a$
    \item $(a + b) + c = a + (b + c)$
    \item An identity of addinition $0$, that is, $a + 0 = a$.
    \item An inverse of $a$, denote by $-a$, such that $a + (- a) = 0$
    \item $(ab)c = a(bc)$
    \item $a(b + c) = ab + ac$ and $(a + b)c = ac + bc$
  \end{enumerate}
\end{definition}

We can observe that ring is a group under addition and some
rules about multiplication 
(i.e. multiplication is associative and is left/right distributive over addition).

We use $n \cdot a$ to indicate the "sum" of $n$ $a$
rather than $na$, since $na$ is used by multiplication.

From the definition, we can get some faimilar properties.

\begin{theorem}
  Let $a \ b \ c \in R$ where $R$ is ring, then
  \begin{enumerate}
    \item $a0 = 0a = 0$
    \item $a(-b) = (-a)b = -(ab)$
    \item $(-a)(-b) = ab$
    \item $a(b - c) = ab - ac$ and $(a - b)c = ac - bc$
    \item If $R$ has a unity element $1$, then $(-1)a = -a$
    \item If $R$ has a unity element $1$, then $(-1)(-1) = 1$
  \end{enumerate}
\end{theorem}
\begin{proof}
  Newline please!
  \begin{enumerate}
    \item Let $b \in R$, $a(b + 0) = ab + a0$ and $a(b + 0) = a(b)$.
          Therefore $ab + a0 = ab$, then $a0 = 0$, same for $0a$.
    \item $a(b + (-b)) = ab + a(-b)$ and $a(b + (-b)) = a0 = 0$.
          Therefore $0 = ab + a(-b)$, then $a(-b) = -(ab)$,
          same for $(-a)b$.
    \item $(-a)(-b) = -((-a)b) = -(-(ab)) = ab$
    \item $a(b - c) = a(b + (-c)) = ab + a(-c) = ab + -(ac) = ab - ac$,
          same for $(a - b)c$.
    \item $(-1)a = -(1a) = -a$
    \item $(-1)(-1) = -(-1) = 1$
  \end{enumerate}
\end{proof}

\begin{theorem}
  If a ring has a unity, then it is unique. 
  If a ring element has a multiplicative inverse, then it is unique.
\end{theorem}
\begin{proof}
  Suppose $1$ and $m$ are the unity of some ring,
  then $1m = 1$ and $1m = m$, since they are unity.

  For any ring element $r$, and $a \ b$ are the inverse of $r$,
  then $(ar)b = b$ and $a(rb) = a$, therefore $a = b$.
\end{proof}

\begin{definition}[Subring]
  A subset $S \subseteq R$ is a subring of $R$ if $(S, +, \times)$ is a ring.
\end{definition}

\begin{lemma}[Subring Test]
  A non-empty set $S \subseteq R$ is a subgring of $R$ if:
  \begin{itemize}
    \item For any $a \ b \in S$, $a - b \in S$.
    \item For any $a \ b \in S$, $ab \in S$.
  \end{itemize}
\end{lemma}

\end{document}