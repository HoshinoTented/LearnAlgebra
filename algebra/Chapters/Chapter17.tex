\documentclass[../main.tex]{subfiles}

\setcounter{section}{17}

\begin{document}

\begin{definition}[Irreducible Polynomials]
  Let $D$ an integral domain and $f(x) \in \poly{D}$ where $f(x)$ is 
  neither zero polynomial nor a unit. We say $f(x)$ is irreducible over $D$,
  if $f(x) = g(x)h(x)$ where $g(x) \ h(x) \in \poly{D}$, then one of them is unit.
  A nonzero, nonunit element of $\poly{D}$ is not irreducible over $D$ is called
  reducible over $D$.
\end{definition}

\begin{definition}[Content]
  The content of a non-zero polynomial is the greatest common divisor of the coefficients.
  A primitive polynomial is an element of $\poly{Z}$ with content $1$.
\end{definition}

\begin{theorem}[Gauss's Lemma]
  The product of two primitive polynomials is primitive.
\end{theorem}
\begin{proof}
  This proof comes from textbook.

  Suppose $f(x) = g(x)h(x)$ where $g(x) \ h(x)$ are primitive. If $f(x)$ is not primitive,
  then we denote $n$ as the content of $f(x)$, then $p$ divides $n$ where $p$ is prime.
  Consider $\overline{f}(x) \ \overline{g}(x) \ \overline{h}(x)$, which are polynomials
  with coefficients $\modu p$.

  Then $\overline{f}(x)$ is a zero polynomial in $\poly{Z_p}$ since the content of $f(x)$
  is dividible by $p$. We know $\poly{Z_p}$ is an integral domain since $Z_p$ is an
  integral domain, then either $g(x)$ or $h(x)$ is a zero polynomial,
  which means the content of $g(x)$ or $h(x)$ is dividible by $p$,
  which contradicts to the assumption that $g(x)$ and $h(x)$ are primitive.
\end{proof}

\begin{lemma}
  If $f(x)$ is reducible, then $(n \cdot 1)f(x)$ is reducible where $n$ is a positive integer.
\end{lemma}
\begin{proof}
  Suppose $f(x) = g(x)h(x)$ where both not unit, then $(n \cdot 1)f(x) = (n \cdot 1)g(x)h(x)$.
  We claim $(n \cdot 1)g(x)$ is not unit. If $(n \cdot 1) g(x)$ is an inverse $\overline{g}$, then
  $(n \cdot 1) g(x) \overline{g} = g(x) (n \cdot 1) \overline{g} = 1$, therefore $g(x)$ is a unit.
  (Recall that a polynomial ring is commutative).
\end{proof}

\begin{theorem}
  Let $f(x) \in \poly{Z}$, if $f(x)$ is reducible over $Q$, then it is reducible over $Z$.
\end{theorem}
\begin{proof}
  This proof comes from textbook.
  
  Suppose $f(x) = g(x)h(x)$ where $g(x) \ h(x) \in \poly{Q}$ and both not unit.
  We may suppose $f(x)$ is primitive, otherwise by Lemma 17.1, $nf(x)/n$ is reducible where $n$ is the content of $f(x)$.
  Let $a$ and $b$ are the $\lcm$ of denominators of $g(x)$ and $h(x)$ respectively,
  then $abf(x) = ag(x)bh(x)$. Furthermore, we may divide $ag(x)$ and $bh(x)$ by their
  contents, now $abf(x) = cg^\prime(x)dh^\prime(x)$ where $c \ d$ are
  the contents of $ag(x) \ bh(x)$ respectively.
  We know $g^\prime(x)$ and $h^\prime(x)$ are primitive, so is $g^\prime(x)h^\prime(x)$.
  Then the content of right hand side is $cd$ and left hand side is $ab$.
  Then $f(x) = g^\prime(x)h^\prime(x)$, it is easy to show $g^\prime(x)$ and $h^\prime(x)$
  is not unit.

  Furthermore, those two polynomials have non-zero degrees, 
  if $f(x)$ is primitive, and $\deg g^\prime(x) = \deg cg^\prime(x) = \deg ag(x) = \deg g(x)$,
  which implies $g(x)$ is a unit;
  if $f(x)$ is not primitive, then $f(x)/n$ can be expressed as two polynomials with non-zero degree,
  so is $nf(x)/n$.
\end{proof}

\begin{theorem}
  Let $p$ a prime and $f(x) \in \poly{Z}$ with $\deg f(x) \geq 1$.
  Let $\overline{f}(x)$ be the polynomial in $\poly{Z_p}$ obtained
  from $f(x)$ by reducing all the coefficients of $f(x)$ modulo $p$.
  If $\overline{f}(x)$ is irreducible over $Z_p$ and $\deg \overline{f}(x) = \deg f(x)$,
  then $f(x)$ is irreducible over $Q$.
\end{theorem}
\begin{proof}
  This proof comes from textbook.

  Suppose $f(x)$ is reducible over $Q$, then $f(x)$ is reducible over $Z$.
  Then $f(x) = g(x)h(x)$ where $g(x) \ h(x) \in \poly{Z}$ and both not unit,
  and $\overline{f}(x) = \overline{g}(x) \overline{h}(x)$. We know $\deg f(x) = \deg \overline{f}(x)$,
  then $\deg g(x) = \deg \overline{g}(x)$ and $\deg h(x) = \deg \overline{h}(x)$
  cause reducing coefficients of modulo $p$ doesn't increase the degree.
  Then we know both $\overline{g}(x)$ and $\overline{h}(x)$ are not unit (cause they have non-zero degree),
  then $\overline{f}(x)$ is reducible over $Z_p$.
\end{proof}

\setcounter{theorem}{4}
\begin{theorem}
  \label{Theorem:17.5}
  Let $F$ a field and $p(x) \in \poly{F}$. Then $\cyc{p(x)}$ is a maximal ideal
  in $\poly{F}$ iff $p(x)$ is irreducible over $F$.
\end{theorem}
\begin{proof}
  ~
  \begin{itemize}
    \item $(\Rightarrow)$ If $p(x) = g(x)h(x)$ for some $g(x) \ h(x) \in \poly{F}$,
          we know $\cyc{p(x)}$ is a prime ideal since it is maximal, then one of
          $g(x)$ and $h(x)$ is in $\cyc{p(x)}$. We may suppose $g(x) \in \cyc{p(x)}$,
          then $g(x)$ is either zero or $\deg g(x) \ge \deg p(x)$,
          but $g(x) \neq 0$ and $\deg g(x) \leq \deg p(x)$ (since $p(x) = g(x)h(x)$), 
          therefore $\deg g(x) = \deg p(x)$ and then $\deg h(x) = 0$, which implies $h(x)$
          is a unit.
    \item $(\Leftarrow)$ Suppose $I$ is an ideal that properly contains $\cyc{p(x)}$,
          we know $\poly{F}$ is a principal ideal domain.
          Suppose $I = \cyc{q(x)}$ for some $q(x)$, then we know $p(x)$
          can be expressed by $q(x)r(x)$ for some $r(x)$ since $p(x) \in \cyc{q(x)}$.
          Then one of $q(x)$ and $r(x)$ is unit, if $q(x)$ is unit, then $1 \in \cyc{q(x)}$,
          if $r(x)$ is unit, then $q(x) = p(x)\inv{r}(x)$, which implies $q(x) \in \cyc{p(x)}$,
          it contradict to the assumption that $I$ properly contains $\cyc{p(x)}$.
  \end{itemize}
\end{proof}

\begin{corollary}
  Let $p(x)$ a irreducible polynomial of $\poly{F}$ and $p(x) \mid a(x)b(x)$.
  Then $p(x) \mid a(x)$ or $p(x) \mid b(x)$.
\end{corollary}
\begin{proof}
  By Theorem 17.5, we know $\cyc{p(x)}$ is a maximal ideal, therefore it is a
  prime ideal. The rest of proof is trivial.
\end{proof}

\begin{corollary}
  \label{Corollary:17.2}
  Let $f(x)$ is irreducible polynomial of $\poly{F}$, then $\poly{F}/\cyc{f(x)}$
  is a field.
\end{corollary}
\begin{proof}
  By Theorem \ref{Theorem:17.5} and Theorem 14.4.
\end{proof}

\begin{theorem}[Unique Factorization]
  Every polynomial in $\poly{Z}$ that is not a zero polynomial or a unit
  in $\poly{Z}$ can be expressed in the form
  \begin{center}
    \boxed{b_0 b_1 \dots b_{s - 1} p_0(x) p_1(x) \dots p_{m - 1}(x)}
  \end{center}
  where the $b_i$'s are irreducible polynomials of degree 0
  (In other words, they are primes),
  and $p_i(x)$'s are irreducible polynomials of positive degree.

  Furthermore, if is can be expressed in two ways, then they have the same $s$ and $m$,
  and they have the same $b_i$'s and $p_i$'s with $\pm$ if needed.
  % They may have different sign
\end{theorem}
\begin{proof}
  Let $f(x) \in \poly{Z}$ where $f(x)$ is not a zero polynomial and not
  a unit in $\poly{Z}$. Induction on the degree of $f(x)$.
  \begin{itemize}
    \item Base: we can written $f(x) = f_0$ in the product of primes,
          and we know that is a unique factorization.
    \item Ind: We can always express $f(x)$ in form $cg(x)$ where $c$ is the content of $f(x)$
          when $f(x)$ is not primitive, then $c$ has unique factorization.
          We need to show that $g(x)$ has unique factorization.
          If $g(x)$ is irreducible, then we the only factorization of $g(x)$ is itself.
          If $g(x)$ is reducible, we know there is $p(x) \in \poly{Z}$ such that
          $p(x) \mid g(x)$ and $p(x)$ is irreducible and primitive.
          We know $p(x)$ must be contained in every factorization of $g(x)$ 
          by Corollary 17.1, then by induction hypothesis, $g(x)/p(x)$ has unique factorization
          and $g(x) = p(x) g(x)/p(x)$.
  \end{itemize}
\end{proof}

\end{document}